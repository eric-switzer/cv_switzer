\newcommand{\lastupdated}{June 17, 2023}
\definecolor{light-gray}{gray}{0.60}
\definecolor{dark-gray}{gray}{0.30}
\newcommand{\ltgray}{\textcolor{light-gray}}
\newcommand{\drkgray}{\textcolor{dark-gray}}
%\newcommand{\reshead}[1]{\section*{#1}}
\newcommand{\reshead}[1]{{\bfseries\large {#1}}}
\newcommand{\linkarticle}[3]{{#1} \href{{#2}}{\small \drkgray{#3}}}
\newcommand{\HRule}{\rule{\linewidth}{0.1mm}}
\newenvironment{resitem}%
  {\begin{itemize}[topsep=0pt, partopsep=0pt, itemsep=0pt, parsep=0pt, midpenalty=300]}%
  {\end{itemize}\vspace{0.5cm}}

%------------------------------------------------------------------------------
% address, background, honors
%------------------------------------------------------------------------------
\newcommand{\addressminimal}[1]{
\begin{minipage}[t]{0.5\textwidth}
  \href{http://www.nasa.gov/centers/goddard/}{NASA Goddard Space Flight Center} \\
  Observational Cosmology Laboratory (665) \\
  Building 34, room E332 \\
  Greenbelt, MD 20771
\end{minipage}
\begin{minipage}[t]{0.5\textwidth}
  {\tt eric.r.switzer at nasa.gov} \\
  Office: 301-614-5174 \\
  Cell: 240-547-8963 \\
  %\href{https://github.com/eric-switzer}{\tt https://github.com/eric-switzer } \\
  Updated: {#1}
\end{minipage}\vspace{0.5cm}
}

\newcommand{\background}[1]{
\begin{resitem}
\item \reshead{Employment and Education}
\item \href{http://www.nasa.gov/centers/goddard/}{Astrophysicist, NASA GSFC, Observational Cosmology Laboratory, 2013-}
\item
\item \href{http://cita.utoronto.ca/}{Senior Research Associate, CITA, University of Toronto, 2011-2013.}
\item \href{http://kicp.uchicago.edu/}{Postdoctoral Fellow, Kavli Institute for Cosmological Physics, University of Chicago, 2008-2011.}
\item \href{http://www.princeton.edu/}{Ph.D. Physics, Princeton University, 2008.}
\item \href{http://www.uchicago.edu}{B.A. Physics (with honors), University of Chicago, 2003.}
%\item Rolla High School, Rolla Missouri
\end{resitem}

Dr. Eric Switzer has over 15 years of experience in research related to
observations of cosmic background radiation from radio to far-IR wavelengths.
His work spans all phases of the project life cycle through concept, design,
integration and test, analysis, and data archival. His scientific interests
include redshifted line intensity mapping, cosmic microwave background
anisotropies, and systems engineering to establish the requirements and design
for next-generation missions. Dr. Switzer is PI of the EXCLAIM mission, and led
cryogenic receiver and flight software development for the PIPER mission.
Before starting at NASA, he led the analysis of 21 cm intensity mapping data
acquired with the Green Bank Telescope, worked to develop and commission the
Atacama Cosmology Telescope, and developed an accurate theory of cosmological
helium recombination.\\

%Dr. Switzer's research areas include CMB temperature and polarization
%anisotropies, redshifted line tomography, atomic physics in cosmological
%settings, and extragalactic sources detected in mm-wavelengths. Eric was part
%of the team that developed the Atacama Cosmology Telescope. At CITA, Eric led
%the analysis of $21$\,cm intensity maps acquired at the Green Bank Telescope.
%At GSFC he has led receiver and software development for PIPER, and is PI of
%the EXCLAIM mission.\\

%Past and present research areas: instrumentation and data analysis for cosmic
%microwave background (CMB) experiments, cosmological helium recombination and
%reionization, mapping diffuse 21 cm emission at $z\sim 1$. Dissertation
%adviser: Lyman Page.  Title: {\it Small-scale anisotropies of the cosmic
%microwave background: Experimental and theoretical perspectives}.\\
}

\newcommand{\comptonhonors}{Compton Lecture Series, ``The Physics of Energy Devices," University of Chicago, fall 2009.}
\newcommand{\comptonhonorshttp}[1]{\linkarticle{\comptonhonors}{http://cita.utoronto.ca/~eswitzer/compton/}{#1}}

\newcommand{\honors}[1]{
\begin{resitem}
\item \reshead{Honors and Awards}
\item 2022 Robert H. Goddard Award for Mentoring
% distinguished 2023, 6/23
% distinguished 2022, 6/22
\item 2021 NASA Special Act and Goddard Science and Exploration Directorate Award for Mentorship
% distinguished 2021, 7/21
\item 2021 NASA Special Act Award: For leadership of the successful PDR for the EXCLAIM mission
% distinguished 2020, 8/20
\item 2020 NASA Special Act Award: Mentorship (students lead roles in 7 EXCLAIM reviews)
% distinguished 2019, 7/19
\item 2019 NASA Special Act Award: In recognition of EXCLAIM APRA success
% Five-year service award
% 2018 distinguished
\item 2017 NASA Special Act Award (Team): PIPER mission engineering flight
% 2017 distinguished
% 2016 distinguished, 8/16
% 2015 distinguished, 7/15
\item 2015 NASA Special Act Award (Team): LAMBDA contributions to HEASARC Senior Review
% ACT$ CMB lensing
\item 2015 APS, ACT Recognized among 32 most influential papers in General Relativity Centennial
% Exceeds performance 2014 (started work)
\item 2014 NASA Special Act Award: Lead contributions to PIPER detector readout and flight software
\item CITA Senior Research Associate, 2011-2013
\item Compton Lecture Series, ``The Physics of Energy Devices," University of Chicago, fall 2009
%\item \comptonhonorshttp{#1}
\item KICP Postdoctoral Fellowship, University of Chicago, 2008-2011
\item Centennial Fellowship, Princeton University Graduate School, 2003-2008
\item Joseph Henry Merit Prize, Princeton University Graduate School, 2003
\item Member of Phi Beta Kappa, University of Chicago
\item DESY-Zeuthen Summer Student Research Fellowship, 2002
\end{resitem}
}

\newcommand{\professional}[1]{
\begin{resitem}
\item \reshead{Professional Activities and Collaborations}
\item 2022-      Roman Space Telescope Wide Field Instrument Integration and Test Project Scientist
\item 2021       Deputy PI: Primordial Inflation Explorer (PIXIE) Proposal
\item 2019-      PI: Experiment for Cryogenic Large-Aperture Intenisty Mapping (EXCLAIM)
\item 2013-      Deputy PI: Primordial Inflation Polarization Explorer (PIPER)
\item 2015-2022      Project Scientist: Legacy Archive for Microwave Background Data (LAMBDA)
\item 2011-2013  Atacama Cosmology Telescope (ACTPol) Collaborator
\item 2011-2013  Lead: Green Bank Telescope $21$\,cm intensity mapping analysis
\item 2008-2011  South Pole Telescope (SPT, Point sources) Collaborator
\item 2005-2011  Atacama Cosmology Telescope (ACT, Instrument), Instrumentation
\end{resitem}
}

\newcommand{\fieldwork}[1]{
\begin{resitem}
\item \reshead{I\&T, commissioning, and field  work}
\item Field work for instrument commissioning and balloon flights (1-2 months typical)
\item 2019-     PIPER engineering flight 2: Ft. Sumner, NM 2019
\item 2018-     PIPER flight attempt 2: Ft. Sumner, NM 2018
\item 2017-     PIPER engineering flight 1: Ft. Sumner, NM 2017
\item 2017-     PIPER flight attempt 2: Palestine, TX
\item 2016-     PIPER flight attempt 1: Ft. Sumner, NM
\item 2008-     ACT MBAC receiver commissioning: San Pedro de Atacama, Chile
\item 2007-     ACT receiver commissioning, first light: San Pedro de Atacama, Chile
\item 2006-     ACT telescope commissioning: Port Coquitlam, British Columbia
\end{resitem}
}

\newcommand{\propreview}[1]{
~\\ \reshead{Proposal Reviews}\\~\\
\begin{tabular}{l p{12cm}} %\multicolumn{2}{l}
Mar. 2023 & Review: LEM X-ray mission cryogenic receiver concept \\
July 2022 & Review: Internal R\&D Step 2 committee \\
Dec. 2021 & Review: APRA (TES readout) Red Team (GSFC) \\ % Rostem TDM
Dec. 2021 & Review: APRA (Continuous ADR) Red Team (GSFC) \\ % Jahromi CADR
Dec. 2020 & Review: APRA (CMB detectors) Red Team (GSFC) \\ % CMBPol
June 2020 & Review: APRA (Mid-IR detectors) Red Team (GSFC) \\ % HIRMES
Apr. 2020 & Service: Canadian NSERC proposal reviewer \\
Apr. 2020 & Service: Deutsche Forschungsgemeinschaft proposal reviewer \\
Jan. 2020 & Service: CETUS decadal response review \\
Apr. 2019 & Review: UK STFC proposal \\
Mar. 2019 & Review: APRA (mm-wave optics) Red Team (GSFC) \\ % TEH Aerogel
%May 2018 & Service: GUSTO SRB \\
Feb. 2018 & Review: APRA (BETTII 2) Red Team (GSFC) \\
Apr. 2017 & Review: XARM detector and readout review \\
Aug 2016 & Review: Internal R\&D Step 2 \\
May 2016 & Review: SSERVI Red Team (GSFC) \\
Feb. 2016 & Review: APRA (BETTII) Red Team (GSFC) \\
July 2015 & Review: WFIRST SIT Red Team (GSFC) \\
July 2014 & Review: DARE Blue Team (Ames) \\
\end{tabular} \\~\\~\\
}

\newcommand{\teachingservice}[1]{
~\\ \reshead{Teaching, Service and Outreach}\\~\\
\begin{tabular}{l p{12cm}} %\multicolumn{2}{l}
Overall & Service: Letters of reference for 28 students/postdocs, or award nominations \\
June 2023 & Service: Roman Space Telescope integration and test and weak lensing postdoc hiring committees \\
June 2023 & Review: JHU thesis defense (external reviewer): Gabriela Sato-Polito \\
Dec. 2023 & PREP committee and defense for Jacob Nellis (cryogenic branch) \\
Oct. 2022 & SOC for Munich Intensity Mapping Workshop \\
Apr. 2022 & Review: UMD thesis defense committee (external advisor): Carolyn Volpert \\
Apr. 2022 & Review: UW-Madison thesis defense committee (external advisor): Trevor Oxholm \\
May. 2021 & Service: Symposium Organizing Committee Line Intensity Mapping (LIM) Workshop \\
Aug. 2020 -- & Service: Detector division/Cosmology coordination committee (GSFC) \\
%Nov. 2020 & Service: Letters of reference for 5 students/postdocs \\
Apr. 2020 & Service: UMD 2nd year project committee member (external): Carrie Volpert \\
%Jan. 2020 & Service: Letters of reference for 5 former students \\
Jun. 2019 & Service: Far-IR staff hiring committee \\
Jan. 2017-Dec 2018 & Service: Astrophysics Division Colloquium (Chair) \\
July 2018 & Service: JHU thesis defense committee (external): Duncan Watts \\
Mar. 2017 & Service: JHU thesis defense committee (external): Patrick Breysse \\
2015-2017 & Service: Astrophysics Division Colloquium (Observational Cosmology) \\
Dec. 2016 & Service: CMB staff hiring committee \\
Oct. 2015 & Service: JHU thesis defense committee (external advisor): Justin Lazear \\
May 2012 & Organizer: ``$21$~cm intensity mapping analysis" workshop, CITA. \\
Nov. 2012 & CITA Postdoc hiring committee \\
Mar. 2011 & Interviews in popular press for ``The Steppenwolf: A proposal for a habitable planet in interstellar space." \\
Mar. 2010  & Early-stage textbook review, ``The Physics of Energy" Jaffe and Taylor, Cambridge University Press. \\
2009--2010 year & KICP Friday seminar committee. \\
Oct. 2009 -- Dec. 2009 & ``The Physics of Energy Devices," Compton Lecture Series, University of Chicago (10 public lectures). \\
May 2009 issue & Interview: George Musser, ``Spectral Sensation," {\it Scientific American}. \\
2008--2010     & Founder and organizer of the energy technology student group within the University of Chicago physics department. \\
2007--         & Referee: {\it The Astrophysical Journal}, MNRAS, JOAA, Class. Quantum Grav., Nature \\
May 2006 -- Sep. 2006 & Developed physics teaching guide for the McGraw Center for Teaching and Learning at Princeton. \\
Sep. 2005 -- Dec. 2005 & Introductory Integrated Engineering/Math/Physics, Problem session (Teaching Assistantship), Princeton. \\ % 191
Feb. 2005 -- May 2005 & Introductory Engineering Physics, Supplemental problem sessions (Teaching Assistantship), Princeton. \\ %104
Sep. 2004 -- May 2005 & Introductory General Physics, Labs (Teaching Assistantship), Princeton. \\ %101, 102
\end{tabular} \\~\\~\\
}

% mentor to Kiyo Masui, Chris ?, Yi-Chao Li, Marzieh Farhng, Liviu Calin
\newcommand{\advising}[1]{
~\\ \reshead{Advising}\\~\\
\begin{tabular}{l p{12cm}} %\multicolumn{2}{l}
Summer 2023 & NASA interns (EXCLAIM): Danny Chmaytelli \\
Fall 2022 & NASA interns (EXCLAIM): Nicole Leung \\
Summer 2022 & NASA interns (EXCLAIM): Diego Suazo De la Rosa, Nicole Leung \\
Spring 2022 & NASA interns (EXCLAIM): Joseph Watson (T. Essinger-Hileman joint) \\
Fall 2021 & NASA interns (EXCLAIM): Nicholas Armbrust, Diego Suazo De la Rosa, Joseph Watson (T. Essinger-Hileman joint) \\
Summer 2021 & NASA interns (EXLCAIM): Tyler Cascalho Cox, Gina Pantano (Trevor Oxholm, joint), Sarah Stewart and Joseph Watson (T. Essinger-Hileman joint), Angelo Gannon (Tatsat Parekh primary) \\
Fall 2019 -- & UW Madison Graduate Student, EXCLAIM (UMD Faculty Advisor: P. Timbie): Trevor Oxholm \\
Spring 2019 -- & UMD Graduate Student, EXCLAIM (UMD Faculty Advisor: A. Bolatto): Carrie Volpert \\
Spring 2018 -- & JHU Postdoc, Intensity Mapping: Chris Anderson \\
Spring 2021 & NASA interns (EXCLAIM): Trevian Jenkins (G. Cataldo joint), Mathias Ramirez, Alberto Martinez, Jim Foquet (G. Cataldo primary). \\
Fall 2020 & NASA interns (EXCLAIM): Justin Trenkamp (T. Essinger-Hileman joint), Nina Ong (E. Barrentine primary), Trevian Jenkins (G. Cataldo joint), UMBC Mechanical Engineering capstone, team of 6 (G. Cataldo joint). \\
Summer 2020 & NASA interns (EXCLAIM): Holly Bennett, Chace Cho (E. Barrentine joint), Joaquin Matticoli, Florian Roselli (G. Cataldo primary), Jared Termini (E. Barrentine, T. Essinger-Hileman primary) \\
Spring 2020 & NASA interns (EXCLAIM): Lee Roger Chevres Fernandez, Gedalia Koehler (E. Barrentine joint), Konrad Shire, Akhil Singareddy \\
2019-2020 & NASA/UMD intern (gap year, EXCLAIM): Jonas Mugge-Durum \\
Summer 2019 & NASA interns (EXCLAIM): Alex Lamb (E. Barrentine, joint), Henry Grant (T. Essinger-Hileman joint) \\
Fall 2016- & NSTR fellow, FTS technology (Primary advisor: J. McMahon): Taylor Baildon (UMich) \\
2015--2018 & JHU Postdoc, PIPER software (Co-advisor, PIPER): Natalie Gandilo \\
2016--2018 & NPP Postdoc, PIPER hardware: Rahul Datta \\
2015--2018 & UMD Graduate student (Co-advisor, PIPER): Sam Pawlyk \\
2013--2015 & JHU Graduate student (Co-advisor, PIPER): Justin Lazear \\
Fall 2014 & OSSI, Undergraduate (PIPER): Mitesh Amin \\
Summer 2014 & USNA-GSFC Exchange, Undergraduate (analysis): Tyler Dickenson \\
Summer 2013 & CITA Undergraduate: Valentin Goblot \\
2012--2013 & CITA Undergraduate, research course/summer: Marat Mufteev \\
2011--2012 & CITA Masters project: Adam Lewis \\
\end{tabular} \\~\\~\\
}


% PIXIE IRAD
% check PIPER APRA
\newcommand{\grants}[1]{
~\\ \reshead{Awarded grants}\\~\\
\begin{tabular}{l p{12cm}}
% Barrentine PI
Oct. 2022, Co-I & APRA, {\em $\mu$-Spec Integrated Spectrometers for Far-Infrared Spectroscopy in Space} (PI: Emily Barrentine) \\
% Timbie PI
Oct. 2022, Co-I & APRA, {\em Modeling and Testing Kinetic Inductance Detectors (KIDs) for Astrophysics} (PI: Peter Timbie) \\
% 100k study funds, Glenn PI
Oct. 2022, Co-I & APRA study, {\em BEGINS: The Balloon Experiment for Galactic INfrared Science} (PI: Jason Glenn) \\
% Anthony Pullen PI 494k, unfunded Co-I
Aug. 2021, Co-I & NSF Astronomgy, {\em Large-Scale Statistics from Line Intensity Mapping Simulations} (PI: Anthony Pullen) \\
% 7.14 M$
Aug. 2018 (5 yr), PI & NASA Astrophysics Research and Analysis Program (APRA) {\em Experiment for Cryogenic Large-aperture Intensity Mapping} (EXCLAIM) \\
% 800k/yr
Jan. 2017 (3 yr) Co-I & NASA Strategic Astrophysics Technology (SAT), {\em High-Efficiency Continuous Cooling for Cryogenic Instruments and sub-Kelvin Detectors} (PI: James Tuttle) \\
% 100k/yr
Jan. 2017 (2.5 yr) PI & NASA Astrophysics Data Analysis Program (ADAP), {\em Constraining star formation through redshifted CO and CII emission in archival CMB data} \\
Jan. 2016 (2.5 yr) Co-I & NASA Astrophysics Data Analysis Program (ADAP), {\em Predicting the sky from 30 MHz to 800 GHz: the extended Global Sky Model} (PI: Adrian Liu) \\
Oct. 2015 (5 yr) Co-I & NASA Astrophysics Research and Analysis Program (APRA) {\em Primordial Inflation Polarization Explorer (PIPER) -- Phase 2} (PI: Alan Kogut) \\
% 0.5 FTE, 2k Led to Switzer 2017, EXCLAIM APRA, CO/CII in archival CMB ADAP.
Oct. 2015 (1 yr), PI & NASA Strategic Innovation Fund (SIF), {\em Mapping the history of the universe with unresolved cosmologically redshifted line radiation}. \\
% 500k/yr, 3yr
Oct. 2015 (4 yr), Co-I & NASA Archives Review, {\em Legacy Archive for Microwave Background Data Analysis (LAMBDA)} (PI: Alan Smale, HEASARC)\\
% GBT/12A-418; lead author but not PI, 300 hr total? Led to GBT 2013 papers.
% https://science.nrao.edu/science/science-program/programs2011a
% http://library.nrao.edu/proposals/catalog/6153
July 2011 (1 yr) & GBT Large program (100 hr/semester) {\em Baryon Acoustic Oscillations with 21cm Intensity Mapping}. \\
Sept. 2011 (2 yr) & CITA Senior Research Associate, University of Toronto (5 year) \\
Oct. 2009 (3 yr) & KICP Postdoctoral Fellow, University of Chicago
\end{tabular} \\~\\~\\
}
% PIXIE IRAD
% check PIPER APRA
\newcommand{\iradgrants}[1]{
~\\ \reshead{Awarded GSFC R\&D grants}\\~\\
\begin{tabular}{l p{12cm}}
% Anh Lah PI 30k and 0.8 FTE
Oct. 2022, Co-I & {\em Laboratory demonstration of quantum-noise-limited amplifiers for future instruments at both GHz and mm bands} (PI: Anh Lah) \\
% Emily Barrentine PI 71k and 1 FTE
Oct. 2022, Co-I & {\em Advancing $\mu$‐Spec spectrometer on‐a‐chip technology for next-generation missions} (PI: Emily Barrentine) \\
% 1.0 FTE, 30k procurement, mid-year
Oct. 2021, PI & {\em Readout development for the PRIMA mission (mid-year)} \\
% 1.05 FTE, 30k procurement
Oct. 2021, PI & {\em Low-background test capabilities for mm-wave to mid-IR astrophysics missions} \\
% 1 FTE 0 procurement
Oct. 2020, PI & {\em  Assessing kinetic inductance detectors for future missions } \\
% 1.1 FTE and 35k + 0.8 FTE and 35k
%FY19:  Microwave multiplexing for future large-format bolometer arrays  $35.5K and 1.4 FTE. This supported a successful SAT (564) for wide-bandwidth DSPs for future flight applications. It spun off an electronics box design used by 1) OST detector development, 2) CMB detector development, 3) the EXCLAIM mission.
%FY20:  Frequency domain multiplexing for future cryogenic detector arrays $35K and 0.8 FTE.
Oct. 2018 (2 yr), PI & {\em Readouts for microwave multiplexing} \\
% 0.1 + 0.8 + 35k
%FY18:  Balloon roadmapping - partnership with USRA $10K and 0.1 FTE, This developed atmospheric emission models that were the basis for EXCLAIM and developed a concept for a next-generation cryogenic balloon.
Jun. 2018 (1.5 yr), PI & {\em Far-IR balloon concept development (CUBIST)} \\
% 15k + 0.4 FTE
%FY18: SMEX development, The signal simulations formed the basis for the EXCLAIM forecast.
Oct. 2017 (1 yr), PI & {\em Concept development for intensity mapping (LIME)} \\
Oct. 2017 (3 yr), Co-I & {\em Balloon concept demonstrator (BOBCAT)} (PI: Alan Kogut) \\
%FY17:  Architectures for Astronomical Hyperspectral Surface Brightness Mapping $10.5K and 0.2 FTE, This developed the software that was used for forecasting sensitivity of the EXCLAIM APRA balloon mission, which was awarded with FY19 start. It supported the LIME FY18 SMEX mission concept IRAD. It was the basis for two publications in 2019.
Oct. 2016 (1 yr), PI & {\em Intensity mapping mission concept development} \\
Oct. 2016 (1 yr) Co-I & {\em PIXIE Technology Test Bed} (PI: Alan Kogut) \\
Oct. 2015 (1 yr) Co-I & {\em PIXIE Instrument Maturation} (PI: Alan Kogut) \\
%FY15:  Architectures and assessment of next-generation CMB polarization instruments $2K and 0.5 FTE. This software was the basis for FY17 and FY18 IRADs, core analysis in two publications (2016, 2017), and contributed to the PIXIE science case in the last MIDEX.
% 68k, 0.6 FTE and 2k$
Oct. 2014 (1 yr), PI & {\em Architectures and Assessment of Next-Generation CMB Polarization Instruments} \\
% 144k, 1.1 FTE and 23k$ Led to 2017 SAT, Tuttle et al. 2017, PIPER CADR flight software.
%FY15:  Achieving High Stability and Efficiency in the Next Generation of Adiabatic Demagnetization Refrigerators $23K and 1.1 FTE. The IRAD supported a successful SAT in the following year to mature Nb3Sn and continuous ADR approaches for future missions. The software and modified electronics for CADR control were used in 2017 and 2019 flights of the PIPER mission. The IRAD effort supported one proceedings and one publication that appeared on the cover of Review of Scientific Instruments.
Oct. 2014 (1 yr), PI & {\em Achieving High Stability and Efficiency in the Next Generation of Adiabatic
 Demagnetization Refrigerators}. \\
\end{tabular} \\~\\~\\
}

\newcommand{\training}[1]{
~\\ \reshead{Training}\\~\\
\begin{tabular}{l p{12cm}}
%TBD 2020 & Safety and Mission Assurance Step 1 (NASA) \\
Feb. 2023 & Detector Systems/Obs. Cosmology Team Dynamics (Higher Echelon) \\
Feb. 2023 & Communicating Technical Issues (NASA APPEL) \\
Feb. 2022 & Creating High Performance Teams (NASA GSFC) \\
Jan. 2022 & Mission Concept Development (NASA GSFC) \\
Sept. 2020 & Flight Projects Development Program, Leadership Workshop (NASA) \\
Nov. 2019 & History of NASA Missions (NASA GSFC) \\
June 2019 & Requirements Development and Management (NASA APPEL) \\
Aug. 2018 & Leadership and Management Skills for non-Managers \\
Aug. 2018 & Speed of Trust -- Foundations (FranklinCovey) \\
July 2018 & Mission Design Workshop (NASA GSFC) \\
July 2018 & Presentation Skills for Technical Professionals (NASA APPEL) \\
Oct. 2017 & Resilience in Leadership (Brookings) \\
Dec. 2016 & Cost and Schedule (NASA Goddard) \\
Oct. 2015 & Capture Planning (NASA Goddard) \\
Aug. 2015 & Leading Through Influence (NASA Goddard) \\
Nov. 2014 & Road to Mission Success (NASA Goddard) \\
Apr. 2014 & Team Leadership (NASA APPEL) \\
Dec. 2013 & NASA Goddard Orientation \\
Recurring safety & ESD operator training, cleanroom, confined spaces, cryogens, oxygen deficiency, ladder safety, CPR, PPE (Eye, face and hand), compressed gases, hazardous waste management, hazard communication, software safety, building emergency plan.
\end{tabular} \\~\\~\\
}

%------------------------------------------------------------------------------
%publication definitions
%------------------------------------------------------------------------------
\newcommand{\ACTsixlensmap}{M.\ Madhavacheril, F.\ Qu, B.\ Sherwin, ACT Collaboration, ``The Atacama Cosmology Telescope: DR6 Gravitational Lensing Map and Cosmological Parameters," {\em Submitted ApJ} (2023).}
\newcommand{\ACTsixlensmaphttp}[1]{\linkarticle{\ACTsixlensmap}{https://ui.adsabs.harvard.edu/abs/2023arXiv230405203M/abstract}{#1}}

\newcommand{\ACTsixlenscosmo}{F.\ Qu, B.\ Sherwin, ACT Collaboration, ``The Atacama Cosmology Telescope: A Measurement of the DR6 CMB Lensing Power Spectrum and its Implications for Structure Growth," {\em Submitted ApJ} (2023).}
\newcommand{\ACTsixlenscosmohttp}[1]{\linkarticle{\ACTsixlenscosmo}{https://ui.adsabs.harvard.edu/abs/2023arXiv230405202Q/abstract}{#1}}

\newcommand{\PIXIEsyst}{A.\ Kogut, D.\ Fixsen, PIXIE Collaboration, ``Systematic error mitigation for the PIXIE Fourier transform spectrometer," {\em Submitted JCAP} (2023).}
\newcommand{\PIXIEsysthttp}[1]{\linkarticle{\PIXIEsyst}{https://ui.adsabs.harvard.edu/abs/2023arXiv230400091K/abstract}{#1}}

\newcommand{\LBOMT}{J.\ Hubmayr, LiteBIRD collaboration, ``Optical Characterization of OMT-Coupled TES Bolometers for LiteBIRD," {\em JLTP}~{\bf 209}(3-4) (2023).}
\newcommand{\LBOMThttp}[1]{\linkarticle{\LBOMT}{https://ui.adsabs.harvard.edu/abs/2022JLTP..209..396H/abstract}{#1}}

% update this
\newcommand{\LBsens}{T.\ Hasebe, LiteBIRD collaboration, ``Sensitivity Modeling for LiteBIRD," {\em JLTP} 10.1007/s10909-022-02921-7 (2023).}
\newcommand{\LBsenshttp}[1]{\linkarticle{\LBsens}{https://ui.adsabs.harvard.edu/abs/2022JLTP..tmp..284H/}{#1}}

% June 2023
\newcommand{\DESlensing}{J.\ Prat, G.\ Zacharegkas, Y.\ Park, N.\ MacCrann, {\bf E.\ R.\ Switzer}, DES Collaboration, ``Non-local contribution from small scales in galaxy-galaxy lensing: Comparison of mitigation schemes," {\em MNRAS}~{\bf 522}(1) (2023).}
\newcommand{\DESlensinghttp}[1]{\linkarticle{\DESlensing}{https://ui.adsabs.harvard.edu/abs/2023MNRAS.522..412P/abstract}{#1}}

\newcommand{\PIPERdetectors}{R.\ Datta, S.\ Dahal, {\bf E.\ R.\ Switzer}, PIPER
collaboration, ``Characterization of Low-noise Backshort-Under-Grid Kilopixel
Transition Edge Sensor Arrays for PIPER," {\em submitted JATIS} (2022).}
\newcommand{\PIPERdetectorshttp}[1]{\linkarticle{\PIPERdetectors}{https://arxiv.org/pdf/2212.01370.pdf}{#1}}

\newcommand{\EXCLAIMforecast}{A.\ R.\ Pullen, P.\ C.\ Breysse, T.\ Oxholm, {\bf
E.\ R.\ Switzer}, EXCLAIM Science Team, ``Extragalactic Science with the Experiment for
Cryogenic Large-aperture Intensity Mapping," {\em MNRAS}~{\bf 521}(4) (2022).}
\newcommand{\EXCLAIMforecasthttp}[1]{\linkarticle{\EXCLAIMforecast}{https://ui.adsabs.harvard.edu/abs/2023MNRAS.521.6124P/abstract}{#1}}

\newcommand{\EXCLAIMSPIECV}{C.\ Volpert, EXCLAIM collaboration, ``Developing a New Generation of Integrated Micro-Spec Far
Infrared Spectrometers for the EXperiment for Cryogenic Large-Aperture Intensity Mapping (EXCLAIM)," {\em Proceedings SPIE} (2022).}
\newcommand{\EXCLAIMSPIECVhttp}[1]{\linkarticle{\EXCLAIMSPIECV}{https://ui.adsabs.harvard.edu/abs/2022arXiv220802786V/abstract}{#1}}

\newcommand{\EXCLAIMSPIEMR}{M.\ Rahmani, EXCLAIM collaboration, ``Optical Characterization \& Testbed Development for $\mu$-Spec
Integrated Spectrometers," {\em Proceedings SPIE} (2022).}

\newcommand{\EXCLAIMSPIETEH}{T.\ Essinger-Hileman, EXCLAIM collaboration, ``EXCLAIM: The EXperiment for Cryogenic Large-Aperture
Intensity Mapping," {\em Proceedings SPIE} (2022).}

\newcommand{\LiteBIRDPTEP}{LiteBIRD Collaboration, ``Probing Cosmic Inflation with the LiteBIRD
Cosmic Microwave Background Polarization Survey," {\em PTEP}~{\bf 150} (2022).}
\newcommand{\LiteBIRDPTEPhttp}[1]{\linkarticle{\LiteBIRDPTEP}{https://doi.org/10.1093/ptep/ptac150}{#1}}

\newcommand{\FIRASBOSS}{C.\ J.\ Anderson, {\bf E.\ R.\ Switzer}, P. C. Breysse,
``Constraining low redshift [C II] Emission by Cross-Correlating FIRAS and BOSS
Data," {\em MNRAS}~{\bf 514}(1) (2022).}
\newcommand{\FIRASBOSShttp}[1]{\linkarticle{\FIRASBOSS}{https://ui.adsabs.harvard.edu/abs/2022MNRAS.514.1169A/abstract}{#1}}

\newcommand{\EXCLAIMdesign}{{\bf E.\ R.\ Switzer}, E.\ M.\ Barrentine, G.\
Cataldo, T.\ Essinger-Hileman, EXCLAIM collaboration, ``Experiment for
Cryogenic Large-Aperture Intensity Mapping: instrument design," {\em J. Astron.
Telesc. Instrum. Syst.}~{\bf 7}(4), 044004 (2021).}
\newcommand{\EXCLAIMdesignhttp}[1]{\linkarticle{\EXCLAIMdesign}{http://dx.doi.org/10.1117/1.JATIS.7.4.044004}{#1}}

% This needs to be updated
\newcommand{\KIDoptimum}{T.\ M.\ Oxholm, {\bf E.\ R.\ Switzer}, E.\ M.\
Barrentine, T.\ Essinger-Hileman, J.\ P.\ Hays-Wehle, P.\ D.\ Mauskopf, A.\ K.\
Sinclair, T.\ R.\ Stevenson, P.\ T.\ Timbie, C.\ Volpert, ``Operational
Optimization to Maximize Dynamic Range in EXCLAIM Microwave Kinetic Inductance
Detectors" {\em JLTP}~{\bf 209}(5-6) (Dec. 2022).}
\newcommand{\KIDoptimumhttp}[1]{\linkarticle{\KIDoptimum}{https://ui.adsabs.harvard.edu/abs/2022JLTP..209.1038O/abstract}{#1}}

\newcommand{\IMCVevade}{T.\ M.\ Oxholm, {\bf E.\ R.\ Switzer}, ``Intensity
mapping without cosmic variance" {\em PRD}~{\bf 104}(8) 083501 (2021).}
\newcommand{\IMCVevadehttp}[1]{\linkarticle{\IMCVevade}{https://ui.adsabs.harvard.edu/abs/2021PhRvD.104h3501O/abstract}{#1}}

% This needs to be updated
\newcommand{\OIIIxCII}{H.\ Padmanabhan, P.\ Breysse, A.\ Lidz, {\bf E.\ R.\
Switzer}, ``Intensity mapping from the sky: synergizing the joint potential of
[OIII] and [CII] surveys at reionization" {\em MNRAS}~stac2025 (2022).}
\newcommand{\OIIIxCIIhttp}[1]{\linkarticle{\OIIIxCII}{https://doi.org/10.1093/mnras/stac2025}{#1}}

\newcommand{\PIPERpumps}{A.\ Kogut, T.\ Essinger-Hileman, {\bf E.\ R.\
Switzer}, E.\ Wollack, D.\ Fixsen, L.\ Lowe, P.\ Mirel, ``Superfluid Liquid Helium Control for the Primordial
Inflation Polarization Explorer Balloon Payload" {\em Rev. Sci. Inst.}~{\bf 92}(064501) (2021).}
\newcommand{\PIPERpumpshttp}[1]{\linkarticle{\PIPERpumps}{https://doi.org/10.1063/5.0048800}{#1}}

\newcommand{\LiteBIRDLFT}{Y.\ Sekimoto, LiteBIRD collaboration, ``Concept
Design of Low Frequency Telescope for CMB B-mode Polarization satellite
LiteBIRD," {\em Proc. SPIE}~{\bf 1145310} (2020).}
\newcommand{\LiteBIRDLFThttp}[1]{\linkarticle{\LiteBIRDLFT}{https://doi.org/10.1117/12.2561841}{#1}}

\newcommand{\LiteBIRDMHFT}{L.\ Montier, LiteBIRD collaboration, ``Overview of the medium and high frequency telescopes of the LiteBIRD space mission
Concept," {\em Proc. SPIE}~{\bf 14432G} (2020).}
\newcommand{\LiteBIRDMHFThttp}[1]{\linkarticle{\LiteBIRDMHFT}{https://doi.org/10.1117/12.2562243}{#1}}

\newcommand{\LiteBIRDoverview}{M.\ Hazumi, LiteBIRD collaboration, ``LiteBIRD
satellite: JAXA's new strategic L-class mission for all-sky surveys of cosmic
microwave background polarization," {\em Proc. SPIE}~{\bf 114432F} (2020).}
\newcommand{\LiteBIRDoverviewhttp}[1]{\linkarticle{\LiteBIRDoverview}{https://doi.org/10.1117/12.2563050}{#1}}

\newcommand{\EXCLAIMSPIEuspec}{M.\ Mirzaei, E.\ Barrentine, EXCLAIM
Collaboration, ``$\mu$-spec spectrometers for the EXCLAIM instrument," {\em
Proc. SPIE}~{\bf 114530M} (2020).}
\newcommand{\EXCLAIMSPIEuspechttp}[1]{\linkarticle{\EXCLAIMSPIEuspec}{https://doi.org/10.1117/12.2562446}{#1}}

\newcommand{\EXCLAIMSPIEoptics}{T.\ Essinger-Hileman, T.\ Oxholm, G.\ Siebert,
EXCLAIM Collaboration, `` Optical Design of the Experiment for Cryogenic
Large-Aperture Intensity Mapping (EXCLAIM)," {\em Proc. SPIE}~{\bf 114530H}
(2020).}
\newcommand{\EXCLAIMSPIEopticshttp}[1]{\linkarticle{\EXCLAIMSPIEoptics}{https://doi.org/10.1117/12.2576254}{#1}}

\newcommand{\EXCLAIMSPIE}{G.\ Cataldo, EXCLAIM Collaboration, ``Overview
and status of EXCLAIM, the experiment for cryogenic large-aperture intensity
mapping," {\em Proc. SPIE}~{\bf 1144524} (2020).}
\newcommand{\EXCLAIMSPIEhttp}[1]{\linkarticle{\EXCLAIMSPIE}{https://doi.org/10.1117/12.2561069}{#1}}

\newcommand{\PIPERwindows}{R.\ Datta, PIPER collaboration,
``Anti-reflection-coated vacuum window for the PIPER balloon-borne instrument,"
{\em Rev. Sci. Inst.}~{\bf 92}(035111) (2021).}
\newcommand{\PIPERwindowshttp}[1]{\linkarticle{\PIPERwindows}{https://doi.org/10.1063/5.0029430}{#1}}

\newcommand{\ACTfourpower}{S.\ K.\ Choi, M.\ Hasselfield, S.\ P.\ Ho, B.\
Koopman, M.\ Lungu, ACT Collaboration, ``The Atacama Cosmology Telescope: A
Measurement of the Cosmic Microwave Background Power Spectra at 98 and 150 GHz"
{\em JCAP}~{\bf 12}:45 (2020).}
\newcommand{\ACTfourpowerhttp}[1]{\linkarticle{\ACTfourpower}{https://ui.adsabs.harvard.edu/abs/2020JCAP...12..045C/abstract}{#1}}

\newcommand{\ACTfoursummary}{S.\ Aiola, E.\ Calabrese, L.\ Maurin, S.\ Naess,
B.\ L.\ Schmitt, ACT Collaboration, ``The Atacama Cosmology Telescope: DR4 Maps
and Cosmological Parameters" {\em JCAP}~{\bf 12}:47 (2020).}
\newcommand{\ACTfoursummaryhttp}[1]{\linkarticle{\ACTfoursummary}{https://ui.adsabs.harvard.edu/abs/2020JCAP...12..047A/abstract}{#1}}

\newcommand{\SPTptsrcfinal}{W.\ Everett, SPT collaboration ``Millimeter-wave
Point Sources from the 2500-square-degree SPT-SZ Survey: Catalog and Population
Statistics" {\em ApJ}(1) (2020).}
\newcommand{\SPTptsrcfinalhttp}[1]{\linkarticle{\SPTptsrcfinal}{https://ui.adsabs.harvard.edu/abs/2020ApJ...900...55E/abstract}{#1}}

\newcommand{\LitebirdLTD}{H.\ Sugai, LiteBIRD collaboration ``Updated
design of the CMB polarization experiment satellite LiteBIRD" {\em JLTP} (2020).}
\newcommand{\LitebirdLTDhttp}[1]{\linkarticle{\LitebirdLTD}{https://arxiv.org/abs/2001.01724}{#1}}

\newcommand{\EXCLAIMoverview}{{\bf E.\ R.\ Switzer}, EXCLAIM Collaboration
``The Experiment for Cryogenic Large-aperture Intensity Mapping (EXCLAIM),"
{\em JLTP} (Jan. 2020).}
\newcommand{\EXCLAIMoverviewhttp}[1]{\linkarticle{\EXCLAIMoverview}{https://arxiv.org/abs/1912.07118}{#1}}

\newcommand{\SPTPolptsrc}{N.\ Gupta, C.\ L.\ Reichardt, SPTPol collaboration,
``Fractional Polarisation of Extragalactic Sources in the 500-square-degree
SPTpol Survey," {\em MNRAS}~{\bf 490} (2019).}
\newcommand{\SPTPolptsrchttp}[1]{\linkarticle{\SPTPolptsrc}{https://ui.adsabs.harvard.edu/abs/2019MNRAS.490.5712G/abstract}{#1}}

\newcommand{\IACuSpec}{G.\ Cataldo, E.\ M.\ Barrentine, B.\ T.\ Bulcha, N.\
Ehsan, L.\ A.\ Hess, O.\ Noroozian, T.\ R.\ Stevenson, E.\ J.\ Wollack, S.\ H.\
Moseley, {\bf E.\ R.\ Switzer}, ``Second-generation Micro-Spec: a compact
spectrometer for far-infrared and submillimeter space missions," {\em Acta Astronautica}~{\bf 162}
(2019).}
\newcommand{\IACuSpechttp}[1]{\linkarticle{\IACuSpec}{https://ui.adsabs.harvard.edu/abs/2019AcAau.162..155C/abstract}{#1}}

\newcommand{\PIPERRX}{{\bf E.\ R.\ Switzer}, PIPER Collaboration, ``Sub-kelvin
cooling for two kilopixel bolometer arrays in the PIPER receiver," {\em Rev. Sci. Inst.}~{\bf 90}(9), Editor's pick/SciLight, {\em cover image} (2019).}
\newcommand{\PIPERRXhttp}[1]{\linkarticle{\PIPERRX}{https://ui.adsabs.harvard.edu/abs/2019RScI...90i5104S/abstract}{#1}}

\newcommand{\OdegardCIB}{N.\ Odegard, J.\ L.\ Weiland, D.\ J.\ Fixsen, D.\ T.\
Chuss, E.\ Dwek, A.\ Kogut, {\bf E.\ R.\ Switzer}, ``Determination of the
Cosmic Infrared Background from COBE/FIRAS and Planck HFI Observations," {\em ApJ}~{\bf 877}(1) (2019).}
\newcommand{\OdegardCIBhttp}[1]{\linkarticle{\OdegardCIB}{http://adsabs.harvard.edu/abs/2019ApJ...877...40O}{#1}}

\newcommand{\PIPERSPIES}{S.\ Pawlyk, PIPER Collaboration, ``The primordial
inflation polarization explorer (PIPER): current status and performance of the
first flight," {\em Proc. SPIE}~{\bf 10708} (2018).}
\newcommand{\PIPERSPIEShttp}[1]{\linkarticle{\PIPERSPIES}{http://adsabs.harvard.edu/abs/2018SPIE10708E..06P}{#1}}


\newcommand{\IMPlanck}{S.\ Yang, A.\ R.\ Pullen, {\bf E.\ R.\ Switzer},
``Evidence for CII diffuse line emission at redshift $z\sim2.6$," {\em MNRASL}~ {\bf slz126} (2019).}
\newcommand{\IMPlanckhttp}[1]{\linkarticle{\IMPlanck}{http://adsabs.harvard.edu/abs/2019arXiv190401180Y}{#1}}

\newcommand{\IMcontinua}{{\bf E.\ R.\ Switzer}, C.\ J.\ Anderson, A.\ Pullen,
S.\ Yang, ``Intensity mapping in the presence of foregrounds and correlated
continuum emission," {\em ApJ}~{\bf 872}(1) (2019).}
\newcommand{\IMcontinuahttp}[1]{\linkarticle{\IMcontinua}{https://doi.org/10.3847/1538-4357/aaf9ab}{#1}}

\newcommand{\ACTpolptsrc}{R.\ Datta, ACTPol Collaboration, ``The Atacama
Cosmology Telescope: Two-season ACTPol Extragalactic Point Sources and their
Polarization Properties," {\em MNRAS}~{\bf 486} (2018).}
\newcommand{\ACTpolptsrchttp}[1]{\linkarticle{\ACTpolptsrc}{https://ui.adsabs.harvard.edu/abs/2019MNRAS.486.5239D/abstract}{#1}}

\newcommand{\ADRSATproc}{J.\ Tuttle, E.\ Canavan, H.\ DeLee, M.\ DiPirro, A.\
Jahromi, B.\ James, M.\ Kimball, P,\ Shirron, D.\ Sullivan, {\bf E.\ Switzer},
``Development of a space-flight ADR providing continuous cooling at 50 mK with
heat rejection at 10 K," Proceedings of Materials Science and Engineering,
Volume 278, Issue 1, pp. 012009 (2017).}
\newcommand{\ADRSATprochttp}[1]{\linkarticle{\ADRSATproc}{http://adsabs.harvard.edu/abs/2017MS&E..278a2009T}{#1}}

\newcommand{\DAREmethod}{K.\ Tauscher, D.\ Rapetti, J.\ O.\ Burns, {\bf E.\ R.\
Switzer}, ``Global 21-cm signal extraction from foreground and instrumental
effects I: Pattern recognition framework for separation using training sets,"
{\em ApJ}~{\bf 853}(2) (2018).}
\newcommand{\DAREmethodhttp}[1]{\linkarticle{\DAREmethod}{http://adsabs.harvard.edu/abs/2018ApJ...853..187T}{#1}}

\newcommand{\Parkescross}{C.\ J.\ Anderson, N.\ J.\ Luciw, Y.-C.\ Li, C.\ Y.\
Yuo, J.\ Yadav, Parkes Collaboration, ``Lack of clustering in low-redshift
21-cm intensity maps cross-correlated with 2dF galaxy densities," {\em
MNRAS}~{\bf 476}:3 (2018).}
\newcommand{\Parkescrosshttp}[1]{\linkarticle{\Parkescross}{http://adsabs.harvard.edu/abs/2018MNRAS.476.3382A}{#1}}

\newcommand{\IMstatus}{E.\ D.\ Kovetz, M.\ P.\ Viero, A.\ Lidz, L.\ Newburgh,
M.\ Rahman, {\bf E.\ R.\ Switzer}, Marc Kamionkowski, et al., ``Line-Intensity
Mapping: 2017 Status Report," 1709.09066 (2017).}
\newcommand{\IMstatushttp}[1]{\linkarticle{\IMstatus}{https://arxiv.org/abs/1709.09066}{#1}}

\newcommand{\CMBSIVtech}{CMB-S4 Technology Contributors, ``CMB-S4 Technology Book, First Edition," 1706.02464 (2017).}
\newcommand{\CMBSIVtechhttp}[1]{\linkarticle{\CMBSIVtech}{https://arxiv.org/abs/1706.02464}{#1}}

\newcommand{\DARErev}{J.\ O.\ Burns, DARE collaboration, ``A Space-Based
Observational Strategy for Characterizing the First Stars and Galaxies Using
the Redshifted 21-cm Global Spectrum," {\em ApJ}~{\bf 844}(1) (2017).}
\newcommand{\DARErevhttp}[1]{\linkarticle{\DARErev}{http://adsabs.harvard.edu/abs/2017ApJ...844...33B}{#1}}

\newcommand{\PIXIEIM}{{\bf E.\ R.\ Switzer}, ``Tracing the cosmological
evolution of stars and cold gas with CMB spectral surveys," {\em ApJ}~{\bf
838}:82, (2017).}
\newcommand{\PIXIEIMhttp}[1]{\linkarticle{\PIXIEIM}{http://adsabs.harvard.edu/abs/2017ApJ...838...82S}{#1}}

\newcommand{\ACTPolStwo}{T.\ Louis, M.\ Hasselfield, M.\ Lungu, L.\ Maurin,
ACTPol collaboration, ``The Atacama Cosmology Telescope: Two-Season ACTPol
Spectra and Parameters," {\em JCAP}~{\bf 6}:31, (2017).}
\newcommand{\ACTPolStwohttp}[1]{\linkarticle{\ACTPolStwo}{http://adsabs.harvard.edu/abs/2016arXiv161002360L}{#1}}

\newcommand{\PIPERSPIEB}{N.\ Gandilo, PIPER Collaboration, ``The Primordial
Inflation Polarization Explorer (PIPER)," {\em Proc. SPIE}~{\bf 9914} (2016).}
\newcommand{\PIPERSPIEBhttp}[1]{\linkarticle{\PIPERSPIEB}{http://adsabs.harvard.edu/abs/2016arXiv160706172G}{#1}}

\newcommand{\CMBlike}{{\bf E.\ R.\ Switzer}, D.\ J.\ Watts, ``Robust
Likelihoods for Inflationary Gravitational Waves from Maps of Cosmic Microwave
Background Polarization," {\em PRD}~{\bf 94} 063526 (2016).}
\newcommand{\CMBlikehttp}[1]{\linkarticle{\CMBlike}{http://journals.aps.org/prd/abstract/10.1103/PhysRevD.94.063526}{#1}}

\newcommand{\GBTICA}{L.\ Wolz, C.\ Blake, F.\ B.\ Abdalla, C.\ M.\ Anderson,
T.-C.\ Chang, Y.-C.\ Li, K.\ W.\ Masui, {\bf E.\ R.\ Switzer}, U.-L.\ Pen, T.\
C.\ Voytek, J.\ Yadav, ``Erasing the Milky Way: new cleaning technique applied
to GBT intensity mapping data," {\em MNRAS}~{\bf 464}(4) (2017).}
\newcommand{\GBTICAhttp}[1]{\linkarticle{\GBTICA}{http://adsabs.harvard.edu/abs/2017MNRAS.464.4938W}{#1}}

\newcommand{\VPMsyst}{N.\ J.\ Miller, D.\ T.\ Chuss, T.\ A.\ Marriage, E.\ J.\
Wollack, J.\ W.\ Appel, C.\ L.\ Bennett, J.\ Eimer, T.\ Essinger-Hileman, D.\
J.\ Fixsen, K.\ Harrington, S.\ H.\ Moseley, K.\ Rostem, {\bf E.\ R.\ Switzer},
D.\ J.\ Watts, ``Recovery of Large Angular Scale CMB Polarization for
Instruments Employing Variable-delay Polarization Modulators," {\em ApJ}~{\bf
818}(2) (2016).}
\newcommand{\VPMsysthttp}[1]{\linkarticle{\VPMsyst}{http://adsabs.harvard.edu/abs/2016ApJ...818..151M}{#1}}

% This produced a significant revision in PAPER results (Ali 2016?)
% described in Cheng 2018 with new results in 1909.02085
\newcommand{\IMmethods}{{\bf E.\ R.\ Switzer}, T.-C.\ Chang, K.\ W.\ Masui,
U.-L.\ Pen, T.\ C.\ Voytek, ``Interpreting the unresolved intensity of
cosmologically redshifted line radiation," {\em ApJ}~{\bf 815}(1), (2015).}
\newcommand{\IMmethodshttp}[1]{\linkarticle{\IMmethods}{http://adsabs.harvard.edu/abs/2015ApJ...815...51S}{#1}}

\newcommand{\ACTPollensing}{A.\ Engelen, B.\ D.\ Sherwin, N.\ Sehgal, ACTPol
Collaboration, ``The Atacama Cosmology Telescope: Lensing of CMB Temperature
and Polarization Derived from Cosmic Infrared Background Cross-correlation,"
{\em ApJ}~{\bf 808}(1) (2015).}
\newcommand{\ACTPollensinghttp}[1]{\linkarticle{\ACTPollensing}{http://adsabs.harvard.edu/abs/2015ApJ...808....7V}{#1}}

% (91531L)
\newcommand{\PIPERSPIEA}{J.\ Lazear, PIPER Collaboration, ``The Primordial
Inflation Polarization Explorer (PIPER)," {\em Proc. SPIE}~{\bf 9153} (2014).}
\newcommand{\PIPERSPIEAhttp}[1]{\linkarticle{\PIPERSPIEA}{http://adsabs.harvard.edu/abs/2014SPIE.9153E..1LL}{#1}}

\newcommand{\EEtoTTreion}{E.\ Calabrese, R.\ Hlozek, N.\ Battaglia, J.\ R.\
Bond, F.\ de Bernardis, M.\ J.\ Devlin, A.\ Hajian, S.\ Henderson, J.\ C.\
Hill, A.\ Kosowsky, T.\ Louis, J.\ McMahon, K.\ Moodley, L.\ Newburgh, M.\ D.\
Niemack, L.\ A.\ Page, B.\ Partridge, N.\ Sehgal, J.\ L.\ Sievers, D.\ N.\
Spergel, S.\ T.\ Staggs, {\bf E.\ R. Switzer}, H.\ Trac, E.\ J.\ Wollack,
``Precision Epoch of Reionization studies with next-generation CMB
experiments," {\em JCAP}~{\bf 08}(010) (2014).}
\newcommand{\EEtoTTreionhttp}[1]{\linkarticle{\EEtoTTreion}{http://adsabs.harvard.edu/abs/2014JCAP...08..010C}{#1}}

\newcommand{\ACTPolEE}{S.\ Naess, M.\ Hasselfield, J.\ McMahon, M.\ Niemack,
ACTPol Collaboration, ``The Atacama Cosmology Telescope: CMB Polarization at
$200<\ell<9000$," {\em JCAP}~{\bf 10}(007) (2014).}
\newcommand{\ACTPolEEhttp}[1]{\linkarticle{\ACTPolEE}{http://adsabs.harvard.edu/abs/2014JCAP...10..007N}{#1}}

\newcommand{\GlobalFG}{{\bf E.\ R.\ Switzer}, A.\ Liu, ``Erasing the
variable: Empirical foreground discovery for global 21 cm spectrum
experiments," {\em ApJ}~{\bf 793}(2) 102 (2014).}
\newcommand{\GlobalFGhttp}[1]{\linkarticle{\GlobalFG}{http://adsabs.harvard.edu/abs/2014ApJ...793..102S}{#1}}

\newcommand{\VPMReview}{D.\ T.\ Chuss, J.\ R.\ Eimer, D.\ J.\ Fixsen, J.\
Hinderks, A.\ J.\ Kogut, J.\ Lazear, P.\ Mirel, {\bf E.\ R.\ Switzer}, G.\ M.\
Voellmer, E.\ J.\ Wollack, ``Variable-delay Polarization Modulators for
Cryogenic Millimeter-wave Applications," {\em Rev. Sci. Inst.}~{\bf 85}(6) (2014).}
\newcommand{\VPMReviewhttp}[1]{\linkarticle{\VPMReview}{http://adsabs.harvard.edu/abs/2014RScI...85f4501C}{#1}}

\newcommand{\ACTlensxoptlens}{N.\ Hand, A.\ Leauthaud, S.\ Das, B.\ D. Sherwin,
ACT Collaboration, ``First Measurement of the Cross-Correlation of CMB Lensing
and Galaxy Lensing," {\em PRD}~{\bf 91}(6) 062001 (2015).}
\newcommand{\ACTlensxoptlenshttp}[1]{\linkarticle{\ACTlensxoptlens}{http://adsabs.harvard.edu/abs/2015PhRvD..91f2001H}{#1}}

\newcommand{\ACTSZradio}{M.\ B.\ Gralla, D.\ Crichton, T.\ A.\ Marriage, W.\
Mo, ACT Collaboration, ``A Measurement of the Millimeter Emission and the
Sunyaev-Zel'dovich Effect Associated with Low-Frequency Radio Sources,"
{\em MNRAS}~{\bf 445}(1) 460 (2014).}
\newcommand{\ACTSZRadiohttp}[1]{\linkarticle{\ACTSZradio}{http://adsabs.harvard.edu/abs/2014MNRAS.445..460G}{#1}}

\newcommand{\SPTptsrctwo}{L.\ M.\ Mocanu, T.\ M.\ Crawford, J.\ D.\ Vieira, SPT
Collaboration, ``Extragalactic millimeter-wave point source catalog, number
counts and statistics from 771 square degrees of the SPT-SZ Survey," {\em
ApJ}~{\bf 779}(1) 61 (2013).}
\newcommand{\SPTptsrctwohttp}[1]{\linkarticle{\SPTptsrctwo}{http://adsabs.harvard.edu/abs/2013ApJ...779...61M}{#1}}

\newcommand{\ACTtwobandptsrc}{D.\ Marsden, M.\ Gralla, T.\ A.\ Marriage, {\bf
E.\ R.\ Switzer}, B.\ Partridge, M.\ Massardi, G.\ Morales, ACT Collaboration,
``The Atacama Cosmology Telescope: Dusty Star-Forming Galaxies and Active
Galactic Nuclei in the Southern Survey," {\em MNRAS}~{\bf 439}(2) 1556 (2014).}
\newcommand{\ACTtwobandptsrchttp}[1]{\linkarticle{\ACTtwobandptsrc}{http://adsabs.harvard.edu/abs/2014MNRAS.439.1556M}{#1}}

\newcommand{\GBTautopower}{{\bf E.\ R.\ Switzer}, K.\ W.\ Masui, K.\ Bandura,
L.-M.\ Calin, T.-C.\ Chang, X.-L.\ Chen, Y.-C.\ Li, Y.-W.\ Liao, A.\ Natarajan,
U.-L.\ Pen, J.\ B.\ Peterson, J.\ R.\ Shaw, T.\ C.\ Voytek, ``Determination of
$z\sim 0.8$ neutral hydrogen fluctuations using the 21 cm intensity mapping
auto-correlation," {\em MNRASL}~{10.1093/slt074} (2013).}
\newcommand{\GBTautopowerhttp}[1]{\linkarticle{\GBTautopower}{http://mnrasl.oxfordjournals.org/content/early/2013/06/19/mnrasl.slt074}{#1}}

\newcommand{\SZmoments}{J.\ Chluba, {\bf E.\ R.\ Switzer}, D.\ Nagai, K.\
Nelson, ``Sunyaev-Zeldovich signal processing and temperature-velocity moment
method for individual clusters," {\em MNRAS}~{\bf 430}(4) 3054 (2013).}
\newcommand{\SZmomentshttp}[1]{\linkarticle{\SZmoments}{http://adsabs.harvard.edu/abs/2013MNRAS.430.3054C}{#1}}

\newcommand{\ACTSZrichness}{N.\ Sehgal, ACT Collaboration, ``The Atacama
Cosmology Telescope: Relation between Galaxy Cluster Optical Richness and
Sunyaev-Zel'dovich Effect," {\em ApJ}~{\bf 767}(1) 38 (2013).}
\newcommand{\ACTSZrichnesshttp}[1]{\linkarticle{\ACTSZrichness}{http://adsabs.harvard.edu/abs/2013ApJ...767...38S}{#1}}

\newcommand{\actfinalbeam}{M.\ Hasselfield, K.\ Moodley, ACT Collaboration,
``The Atacama Cosmology Telescope: Beam Measurements and the Microwave
Brightness Temperatures of Uranus and Saturn," {\em ApJS}~{\bf 209}(1) 17
(2013).}
\newcommand{\actfinalbeamhttp}[1]{\linkarticle{\actfinalbeam}{http://adsabs.harvard.edu/abs/2013ApJS..209...17H}{#1}}

\newcommand{\ACTcluseq}{F.\ Menanteau, C.\ Sif\'{o}n, ACT Collaboration, ``The
Atacama Cosmology Telescope: Physical Properties of Sunyaev-Zel'dovich Effect
Clusters on the Celestial Equator," {\em ApJ}~{\bf 765}(1) 67, (2013).}
\newcommand{\ACTcluseqhttp}[1]{\linkarticle{\ACTcluseq}{http://adsabs.harvard.edu/abs/2013ApJ...765...67M}{#1}}

\newcommand{\preplanck}{E.\ Calabrese, R.\ Hlozek, ACT Collaboration,
"Cosmological parameters from pre-Planck cosmic microwave background
measurements," {\em PRD}~{\bf 87}(10), 103012 (2013).}
\newcommand{\preplanckhttp}[1]{\linkarticle{\preplanck}{http://adsabs.harvard.edu/abs/2013PhRvD..87j3012C}{#1}}

\newcommand{\actfinalpwr}{S.\ Das, T.\ Louis, M.\ R.\ Nolta, ACT Collaboration,
``The Atacama Cosmology Telescope: Temperature and Gravitational Lensing Power
Spectrum Measurements from Three Seasons of Data," {\em JCAP}~{\bf 4} 14, (2014).}
\newcommand{\actfinalpwrhttp}[1]{\linkarticle{\actfinalpwr}{http://adsabs.harvard.edu/abs/2014JCAP...04..014D}{#1}}

\newcommand{\actfinallike}{J.\ Dunkley, E.\ Calabrese, J.\ Sievers, ACT
Collaboration, ``The Atacama Cosmology Telescope: likelihood for small-scale
CMB data," {\em JCAP}~{\bf 7}(25), (2013).}
\newcommand{\actfinallikehttp}[1]{\linkarticle{\actfinallike}{http://adsabs.harvard.edu/abs/2013JCAP...07..025D}{#1}}

\newcommand{\actfinalsz}{M.\ Hasselfield, M.\ Hilton, T.\ A.\ Marriage, ACT
Collaboration, ``The Atacama Cosmology Telescope: Sunyaev-Zel'dovich selected
galaxy clusters at 148 GHz from three seasons of data," {\em JCAP}~{\bf 7}(8),
(2013).}
\newcommand{\actfinalszhttp}[1]{\linkarticle{\actfinalsz}{http://adsabs.harvard.edu/abs/2013JCAP...07..008H}{#1}}

\newcommand{\actfinalparam}{J.\ L.\ Sievers, R.\ A.\ Hlozek, M.\ R.\ Nolta,
ACT Collaboration, ``The Atacama Cosmology Telescope: cosmological parameters
from three seasons of data," {\em JCAP}~{\bf 10}(60), (2013).}
\newcommand{\actfinalparamhttp}[1]{\linkarticle{\actfinalparam}{http://adsabs.harvard.edu/abs/2013JCAP...10..060S}{#1}}

\newcommand{\recombmodes}{M.\ Farhang, J.\ R.\ Bond, J.\ Chluba, {\bf E.\ R.\
Switzer}, ``Constraints on Perturbations to the Recombination History from
Measurements of the Cosmic Microwave Background Damping Tail," {\em ApJ}~{\bf
764}(2) 137 (2013).}
\newcommand{\recombmodeshttp}[1]{\linkarticle{\recombmodes}{http://adsabs.harvard.edu/abs/2013ApJ...764..137F}{#1}}

\newcommand{\ACTkSZxBOSS}{N.\ Hand, ACT Collaboration, ``Detection of Galaxy
Cluster Motions with the Kinematic Sunyaev-Zel'dovich Effect," {\em PRL}~{\bf
109}(4) 041101 (2012).}
\newcommand{\ACTkSZxBOSShttp}[1]{\linkarticle{\ACTkSZxBOSS}{http://adsabs.harvard.edu/abs/2012PhRvL.109d1101H}{#1}}

\newcommand{\GBTxWiggleZ}{K.\ W.\ Masui, {\bf E.\ R.\ Switzer}, N.\ Banavar,
K.\ Bandura, C.\ Blake, L.-M.\ Calin, T.-C.\ Chang, X.\ Chen, Y.-C.\ Li, Y.-W.\
Liao, A.\ Natarajan, U.-L.\ Pen, J.\ B.\ Peterson, J.\ R.\ Shaw, T.\ C.\
Voytek, ``Measurement of $21$\,cm brightness fluctuations at $z\sim 0.8$ in
cross-correlation," {\em ApJL}~{\bf 763}(1) L20 (2013).}
\newcommand{\GBTxWiggleZhttp}[1]{\linkarticle{\GBTxWiggleZ}{http://adsabs.harvard.edu/abs/2013ApJ...763L..20M}{#1}}

\newcommand{\ACTdata}{R.\ D\"{u}nner, M.\ Hasselfield, T.\ A.\ Marriage, J.\
Sievers, ACT Collaboration, ``The Atacama Cosmology Telescope: Data
Characterization and Mapmaking," {\em ApJ}~{\bf 762}(1) 10 (2013).}
\newcommand{\ACTdatahttp}[1]{\linkarticle{\ACTdata}{http://adsabs.harvard.edu/abs/2013ApJ...762...10D}{#1}}

\newcommand{\ACTkxQSO}{B.\ D.\ Sherwin, S.\ Das, A.\ Hajian, ACT Collaboration,
``The Atacama Cosmology Telescope: Cross-Correlation of CMB Lensing and
Quasars," {\em PRD}~{\bf 86}(8) 083006 (2012).}
\newcommand{\ACTkxQSOhttp}[1]{\linkarticle{\ACTkxQSO}{http://adsabs.harvard.edu/abs/2012PhRvD..86h3006S}{#1}}

\newcommand{\HeItrans}{J.\ Chluba, J.\ Fung, {\bf E.\ R.\ Switzer}, ``Radiative
transfer effects during primordial helium recombination," {\em MNRAS}~{\bf
423}(4) 3227 (2012).}
\newcommand{\HeItranshttp}[1]{\linkarticle{\HeItrans}{http://adsabs.harvard.edu/abs/2012MNRAS.423.3227C}{#1}}

\newcommand{\ACTskewness}{M.\ J.\ Wilson, B.\ D.\ Sherwin, J.\ C.\ Hill, ACT
Collaboration, ``Atacama Cosmology Telescope: A measurement of the thermal
Sunyaev-Zel'dovich effect using the skewness of the CMB temperature
distribution," {\em PRD}~{\bf 86}(12) 122005 (2012).}
\newcommand{\ACTskewnesshttp}[1]{\linkarticle{\ACTskewness}{http://adsabs.harvard.edu/abs/2012PhRvD..86l2005W}{#1}}

\newcommand{\ACTxBLAST}{A.\ Hajian, M.\ P.\ Viero, ACT Collaboration,
``Correlations in the (Sub)millimeter background from ACTxBLAST," {\em
ApJ}~{\bf 744}(1) 40 (2012).}
\newcommand{\ACTxBLASThttp}[1]{\linkarticle{\ACTxBLAST}{http://adsabs.harvard.edu/abs/2012ApJ...744...40H}{#1}}

\newcommand{\ACTparam}{J.\ Dunkley, R.\ Hlozek, J.\ Sievers, ACT Collaboration,
``The Atacama Cosmology Telescope: Cosmological Parameters from the 2008 Power
Spectrum," {\em ApJ}~{\bf 739}(1) 52 (2011).}
\newcommand{\ACTparamhttp}[1]{\linkarticle{\ACTparam}{http://adsabs.harvard.edu/abs/2011ApJ...739...52D}{#1}}

\newcommand{\ACTSZA}{E.\ D.\ Reese, T.\ Mroczkowski, F.\ Menanteau, M.\ Hilton,
J.\ Sievers, ACT Collaboration, ``The Atacama Cosmology Telescope:
High-Resolution Sunyaev-Zel'dovich Array Observations of ACT SZE-selected
Clusters from the Equatorial Strip" {\em ApJ}~{\bf 751}(1) 12 (2012).}
\newcommand{\ACTSZAhttp}[1]{\linkarticle{\ACTSZA}{http://adsabs.harvard.edu/abs/2012ApJ...751...12R}{#1}}

\newcommand{\ACTPpower}{R.\ Hlozek, J.\ Dunkley, ACT Collaboration, ``The
Atacama Cosmology Telescope: a measurement of the primordial power spectrum,"
{\em ApJ}~{\bf 749}(1) 90 (2012).}
\newcommand{\ACTPpowerhttp}[1]{\linkarticle{\ACTPpower}{http://adsabs.harvard.edu/abs/2012ApJ...749...90H}{#1}}

\newcommand{\ACTLambda}{B.\ Sherwin, J.\ Dunkley, S.\ Das, ACT Collaboration,
``Evidence for Dark Energy from the Cosmic Microwave Background Alone Using the
Atacama Cosmology Telescope Lensing Measurements," {\em PRL}~{\bf 107}(2)
021302 (2011).}
\newcommand{\ACTLambdahttp}[1]{\linkarticle{\ACTLambda}{http://adsabs.harvard.edu/abs/2011PhRvL.107b1302S}{#1}}

\newcommand{\ACTlens}{S.\ Das, B.\ Sherwin, ACT Collaboration, ``Detection of
the Power Spectrum of Cosmic Microwave Background Lensing by the Atacama
Cosmology Telescope," {\em PRL}~{\bf 107}(2) 021301 (2011).}
\newcommand{\ACTlenshttp}[1]{\linkarticle{\ACTlens}{http://adsabs.harvard.edu/abs/2011PhRvL.107b1301D}{#1}}

\newcommand{\Steppenwolf}{D.\ S.\ Abbot, {\bf E.\ R.\ Switzer}, ``The
Steppenwolf: A Proposal for a Habitable Planet in Interstellar Space," {\em
ApJL}~{\bf 735}(2) L27 (2011).}
\newcommand{\Steppenwolfhttp}[1]{\linkarticle{\Steppenwolf}{http://adsabs.harvard.edu/abs/2011ApJ...735L..27A}{#1}}

\newcommand{\ACTLRG}{N.\ Hand, ACT Collaboration, ``The Atacama Cosmology
Telescope: Detection of Sunyaev-Zel'Dovich Decrement in Groups and Clusters
Associated with Luminous Red Galaxies," {\em ApJ}~{\bf 736}(1) 39 (2011).}
\newcommand{\ACTLRGhttp}[1]{\linkarticle{\ACTLRG}{http://adsabs.harvard.edu/abs/2011ApJ...736...39H}{#1}}

\newcommand{\ACTclustercatalog}{T.\ A.\ Marriage, ACT Collaboration, ``The
Atacama Cosmology Telescope: Sunyaev-Zel'dovich-Selected Galaxy Clusters at
$148$\,GHz in the 2008 Survey," {\em ApJ}~{\bf 737}(2) 61 (2011).}
\newcommand{\ACTclustercataloghttp}[1]{\linkarticle{\ACTclustercatalog}{http://adsabs.harvard.edu/abs/2011ApJ...737...61M}{#1}}

\newcommand{\ACTclustercosmo}{N.\ Sehgal, H.\ Trac, ACT Collaboration, ``The
Atacama Cosmology Telescope: Cosmology from Galaxy Clusters Detected via the
Sunyaev-Zel'dovich Effect," {\em ApJ}~{\bf 732}(1) 44 (2011).}
\newcommand{\ACTclustercosmohttp}[1]{\linkarticle{\ACTclustercosmo}{http://adsabs.harvard.edu/abs/2011ApJ...732...44S}{#1}}

\newcommand{\ACTdaspwr}{S.\ Das, T.\ A.\ Marriage, ACT Collaboration, ``The
Atacama Cosmology Telescope: A Measurement of the Cosmic Microwave Background
Power Spectrum at $148$ and $218$~GHz from the 2008 Southern Survey," {\em
ApJ}~{\bf 729}(1) 62-78 (2011).}
\newcommand{\ACTdaspwrhttp}[1]{\linkarticle{\ACTdaspwr}{http://adsabs.harvard.edu/abs/2011ApJ...729...62D}{#1}}

\newcommand{\ACTcalib}{A.\ Hajian, ACT Collaboration, ``The Atacama Cosmology
Telescope: Calibration with WMAP Using Cross-Correlations," {\em ApJ}~{\bf 740}
86 (2011).}
\newcommand{\ACTcalibhttp}[1]{\linkarticle{\ACTcalib}{http://iopscience.iop.org/0004-637X/740/2/86/}{#1}}

\newcommand{\ACTptsrc}{T.\ A.\ Marriage, J.\ B.\ Juin, Y.-T.\ Lin, D.\ Marsden,
M.\ R.\ Nolta, B.\ Partridge, ACT Collaboration, ``Atacama Cosmology Telescope:
Extragalactic Sources at $148$~GHz in the 2008 Survey," {\em ApJ}~{\bf 731}(2)
100 (2011).}
\newcommand{\ACTptsrchttp}[1]{\linkarticle{\ACTptsrc}{http://adsabs.harvard.edu/abs/2011ApJ...731..100M}{#1}}

\newcommand{\ACTinstrument}{D.\ S.\ Swetz, ACT Collaboration, ``Overview of the
Atacama Cosmology Telescope: Receiver, Instrumentation, and Telescope Systems,"
{\em ApJS}~{\bf 194}(2) 41 (2011).}
\newcommand{\ACTinstrumenthttp}[1]{\linkarticle{\ACTinstrument}{http://adsabs.harvard.edu/abs/2011ApJS..194...41S}{#1}}

\newcommand{\ACTopticalcluster}{F.\ Menanteau, J.\ González, J.\ B.\ Juin, T.\
A.\ Marriage, E.\ D.\ Reese, ACT Collaboration, ``The Atacama Cosmology
Telescope: Physical Properties and Purity of a Galaxy Cluster Sample Selected
via the Sunyaev-Zel'dovich Effect," {\em ApJ}~{\bf 723}(2) 1523-1541 (2010).}
\newcommand{\ACTopticalclusterhttp}[1]{\linkarticle{\ACTopticalcluster}{http://adsabs.harvard.edu/abs/2010ApJ...723.1523M}{#1}}

\newcommand{\HeIforest}{M.\ McQuinn, {\bf E.\ R.\ Switzer}, ``The He{\sc ~i}
$584~$\AA\ Forest as a Diagnostic of Helium Reionization," MNRAS~{\bf 408}(3)
1945-1955 (2010).}
\newcommand{\HeIforesthttp}[1]{\linkarticle{\HeIforest}{http://adsabs.harvard.edu/abs/2010MNRAS.408.1945M}{#1}}

\newcommand{\actpwrspec}{J.\ W.\ Fowler, ACT Collaboration, ``The Atacama
Cosmology Telescope: A Measurement of the $600< \ell <8000$ Cosmic Microwave
Background Power Spectrum at $148$~GHz," {\em ApJ}~{\bf 722}(2) 1148-1161
(2010).}
\newcommand{\actpwrspechttp}[1]{\linkarticle{\actpwrspec}{http://adsabs.harvard.edu/abs/2010ApJ...722.1148F}{#1}}

\newcommand{\sptptsrcclustering}{N.\ R.\ Hall, R.\ Keisler, L.\ Knox, C.\ L.\
Reichardt, SPT Collaboration, ``Angular Power Spectra of the Millimeter
Wavelength Background Light from Dusty Star-forming Galaxies with the South
Pole Telescope," {\em ApJ}~{\bf 718}(2) 632-646 (2010).}
\newcommand{\sptptsrcclusteringhttp}[1]{\linkarticle{\sptptsrcclustering}{http://adsabs.harvard.edu/abs/2010ApJ...718..632H}{#1}}

\newcommand{\sptptsrcmethod}{T.\ M.\ Crawford, {\bf E.\ R.\ Switzer}, W.\ L.\
Holzapfel, C.\ L.\ Reichardt, D.\ P.\ Marrone, J.\ D.\ Vieira, ``A Method for
Individual Source Brightness Estimation in Single- and Multi-band Data," {\em
ApJ}~{\bf 718}(1) 513-521 (2010).}
\newcommand{\sptptsrcmethodhttp}[1]{\linkarticle{\sptptsrcmethod}{http://adsabs.harvard.edu/abs/2010ApJ...718..513C}{#1}}

\newcommand{\sptptsrccatalog}{J.\ D.\ Vieira, T.\ M.\ Crawford, {\bf E.\ R.\
Switzer}, SPT Collaboration, ``Extragalactic Millimeter-wave Sources in South
Pole Telescope Survey Data: Source Counts, Catalog, and Statistics for an 87
Square-degree Field," {\em ApJ}~{\bf 719}(1) 763-783 (2010).}
\newcommand{\sptptsrccataloghttp}[1]{\linkarticle{\sptptsrccatalog}{http://adsabs.harvard.edu/abs/2010ApJ...719..763V}{#1}}

\newcommand{\actbeamsz}{A.\ D.\ Hincks, ACT Collaboration, ``The Atacama
Cosmology Telescope (ACT): Beam Profiles and First SZ Cluster Maps," {\em
ApJS}~{\bf 191}(2) 423-438 (2010).}
\newcommand{\actbeamszhttp}[1]{\linkarticle{\actbeamsz}{http://iopscience.iop.org/0067-0049/191/2/423/}{#1}}

\newcommand{\hethreeplus}{M.\ McQuinn, {\bf E.\ R.\ Switzer}, ``Redshifted
intergalactic $^3{\rm He}+$ 8.7 GHz hyperfine absorption," {\em PRD}~{\bf
80}(6) 063010 (2009).}
\newcommand{\hethreeplushttp}[1]{\linkarticle{\hethreeplus}{http://adsabs.harvard.edu/abs/2009PhRvD..80f3010M}{#1}}

\newcommand{\thesis}{{\bf E.\ R.\ Switzer}, ``Small-scale anisotropies of the
cosmic microwave background: Experimental and theoretical perspectives,'' {\em
Princeton Ph.D. Thesis} (2008).}
\newcommand{\thesishttp}[1]{\linkarticle{\thesis}{http://cita.utoronto.ca/~eswitzer/documents/switzer_thesis.pdf}{#1}}

\newcommand{\recthree}{{\bf E.\ R.\ Switzer}, C.\ M.\ Hirata, ``Primordial
helium recombination III: Thomson scattering, isotope shifts, and cumulative
results,'' {\em PRD}~{\bf 77}(8) 083008 (2008).}
\newcommand{\recthreehttp}[1]{\linkarticle{\recthree}{http://adsabs.harvard.edu/abs/2008PhRvD..77h3008S}{#1}}

\newcommand{\rectwo}{C.\ M.\ Hirata, {\bf E.\ R.\ Switzer}, ``Primordial helium
recombination II: two-photon processes," {\em PRD}~{\bf 77}(8) 083007 (2008).}
\newcommand{\rectwohttp}[1]{\linkarticle{\rectwo}{http://adsabs.harvard.edu/abs/2008PhRvD..77h3007H}{#1}}

\newcommand{\recone}{{\bf E.\ R.\ Switzer}, C.\ M.\ Hirata, ``Primordial helium
recombination I: feedback, line transfer, and continuum opacity," {\em
PRD}~{\bf 77}(8) 083006 (2008).}
\newcommand{\reconehttp}[1]{\linkarticle{\recone}{http://adsabs.harvard.edu/abs/2008PhRvD..77h3006S}{#1}}

\newcommand{\actopt}{J.\ W.\ Fowler, M.\ D.\ Niemack, S.\ R.\ Dicker, ACT
Collaboration, ``Optical design of the Atacama Cosmology Telescope and the
Millimeter Bolometric Array Camera," {\it Applied Optics}~{\bf 46}(17)
3444-3454 (2007).}
\newcommand{\actopthttp}[1]{\linkarticle{\actopt}{http://adsabs.harvard.edu/abs/2007ApOpt..46.3444F}{#1}}

\newcommand{\lisupp}{{\bf E.\ R.\ Switzer}, C.\ M.\ Hirata, ``Ionizing
radiation from hydrogen recombination strongly suppresses the lithium
scattering signature in the CMB," {\em PRD}~{\bf 72}(8) 083002 (2005).}
\newcommand{\lisupphttp}[1]{\linkarticle{\lisupp}{http://adsabs.harvard.edu/abs/2005PhRvD..72h3002S}{#1}}

\newcommand{\neutrinolss}{K.\ Abazajian, {\bf E.\ R.\ Switzer}, S.\ Dodelson,
K.\ Heitmann, S.\ Habib, ``Nonlinear cosmological matter power spectrum with
massive neutrinos: The halo model,'' {\em PRD}~{\bf 71}(4) 043507 (2005).} % Salman, Kevork, Katrin
\newcommand{\neutrinolsshttp}[1]{\linkarticle{\neutrinolss}{http://adsabs.harvard.edu/abs/2005PhRvD..71d3507A}{#1}}

\newcommand{\electrochem}{J.\ A.\ Switzer,  C.-J.\ Hung, L.-Y.\ Huang, {\bf E.\
R.\ Switzer}, T.\ D.\ Golden, and E.\ W.\  Bohannan, ``Electrochemical
Self-Assembly of Copper/Cuprous Oxide Layered Nanostructures," {\em J.\ Am.\
Chem.\ Soc.\ }~{\bf 120} 3530-3531 (1998).}
\newcommand{\electrochemhttp}[1]{\linkarticle{\electrochem}{http://pubs3.acs.org/acs/journals/toc.page?incoden=jacsat&indecade=&involume=120&inissue=14}{#1}}

%------------------------------------------------------------------------------
%reports, proceedings and talks
%------------------------------------------------------------------------------
\newcommand{\HQEXCLAIM}{``EXCLAIM! The EXperiment for Cryogenic Large-Aperture Intensity Mapping," {\em HQ Longwave PI workshop}, November 2022 joint presentation by Tom Essinger-Hileman.}

%Whitepapers Endorsed CMB S4, PICO Author of IM Author/endorser of spectral
%distortion
% https://www.lorentzcenter.nl/site/index.php?pntHandler=WorkshopTemplatePage&pntType=ConPagina&id=1680
\newcommand{\IMspace}{``Line Intensity Mapping for Future Spectro-polarimetric missions," {\em Mission: Spectro-polarimetry of the Microwave Sky. Lorentz Center, Leiden, Netherlands}, October 2022.} % Oct 31

\newcommand{\CHORDcorr}{``Lessons learned for intensity mapping cross-correlation," {\em CHORD workshop}, October 2021.} % Oct 22

\newcommand{\SOFIAEXCLAIM}{``EXCLAIM: a new balloon mission to map the
cosmological history of galaxies," {\em SOFIA Colloquium}, May 2021.} % May 19

\newcommand{\UWEXCLAIM}{``EXCLAIM: a new balloon mission to map the
cosmological history of galaxies," {\em UW-Madison Astronomy Colloquium}, Nov. 2020.} % Nov 5

\newcommand{\SEDEXCLAIM}{``EXCLAIM: a new balloon mission to map the
cosmological history of galaxies," {\em Science and Exploration Directorate
Director's Seminar (GSFC)}, Oct. 2020.} % Oct 8

\newcommand{\PIPERAASDetector}{Poster: R. Datta for PIPER Collaboration, ``The
Primordial Inflation Polarization ExploreR (PIPER): Preflight Characterization
of the Detector Arrays," {\em AAS 236}, June 2020.}
\newcommand{\PIPERAASDetectorhttp}[1]{\linkarticle{\PIPERAASDetector}{https://aas236-aas.ipostersessions.com/default.aspx?s=58-A6-76-14-24-F0-D6-B6-31-90-89-BF-24-D8-B0-D2}{#1}}

\newcommand{\PIPERAASFlight}{Poster: T. Essinger-Hileman for PIPER Collaboration, ``The
Primordial Inflation Polarization ExploreR (PIPER): 2019 Flight and Telescope
Performance," {\em AAS 236}, June 2020.}
\newcommand{\PIPERAASFlighthttp}[1]{\linkarticle{\PIPERAASFlight}{https://aas236-aas.ipostersessions.com/default.aspx?s=E5-6A-03-3B-A4-E4-50-90-82-F1-49-DE-26-42-7D-3A}{#1}}

\newcommand{\PIPERAASMission}{Poster: A. Kogut for PIPER Collaboration, ``The
Primordial Inflation Polarization ExploreR (PIPER): Science Goals," {\em AAS
236}, June 2020.}
\newcommand{\PIPERAASMissionhttp}[1]{\linkarticle{\PIPERAASMission}{https://aas236-aas.ipostersessions.com/default.aspx?s=FB-1C-A9-03-EE-ED-E0-77-81-D7-3E-B6-D6-C2-94-A1}{#1}}

\newcommand{\PIPERAASReceiver}{Poster: E. Switzer for PIPER Collaboration, ``The
Primordial Inflation Polarization ExploreR (PIPER): Receiver Design and
Performance," {\em AAS 236}, June 2020.}
\newcommand{\PIPERAASReceiverhttp}[1]{\linkarticle{\PIPERAASReceiver}{https://aas236-aas.ipostersessions.com/default.aspx?s=60-76-27-4A-07-81-31-C5-22-3D-D0-5F-BE-10-54-84}{#1}}

\newcommand{\EXCLAIMAASMission}{Poster: T. Oxholm for EXCLAIM Collaboration, ``The
Experiment for Cryogenic Large-Aperture Intensity Mapping," {\em AAS 236}, June
2020.}
\newcommand{\EXCLAIMAASMissionhttp}[1]{\linkarticle{\EXCLAIMAASMission}{https://aas236-aas.ipostersessions.com/default.aspx?s=E7-03-19-A9-03-04-B2-91-4A-8B-46-34-21-0A-E0-1F}{#1}}


% also on J.\ Delabrouille, Voyage 2050 whitepaper
% wrote IM section
\newcommand{\spectralvoyage}{J.\ Chluba lead, ``New Horizons in Cosmology with
Spectral Distortions of the Cosmic Microwave Background" Submitted to ESA
Voyage 2050 call for White Papers (2019).}
\newcommand{\spectralvoyagehttp}[1]{\linkarticle{\spectralvoyage}{https://ui.adsabs.harvard.edu/abs/2019arXiv190901593C/abstract}{#1}}

\newcommand{\higateway}{D.\ Rapetti, K.\ Tauscher, J.\ O.\ Burns, E.\ Switzer, J.\ Mirocha, S.\ Furlanetto, R.\ Monsalve, "Hydrogen Cosmology from the Deep Space Gateway: Data Analysis Pipeline for Low-Frequency Radio Telescopes" Deep Space Gateway Concept Science Workshop, proceeedings (2018).}
\newcommand{\higatewayhttp}[1]{\linkarticle{\higateway}{http://adsabs.harvard.edu/abs/2018LPICo2063.3087R}{#1}}

\newcommand{\lambdaprimer}{G.\ E.\ Addison, E.\ R.\ Switzer, M.\ R.\ Greason,
T.\ B.\ Griswold, T.\ Jaffe, N.\ Miller, N.\ P.\ Odegard, U.\ Prasad, J.\ L.\
Weiland, "Legacy Archive for Microwave Background Data Analysis (LAMBDA): An
Overview" (2019).}
\newcommand{\lambdaprimerhttp}[1]{\linkarticle{\lambdaprimer}{http://adsabs.harvard.edu/abs/2019arXiv190508667A}{#1}}

% 1 Astro2020 Science White Paper: Insights Into the Epoch of Reionization with the Highly-Redshifted 21-cm Line, Furlanetto, 2019BAAS...51c.143F, 2019arXiv190306204F
% 2 Fundamental Cosmology in the Dark Ages with 21-cm Line Fluctuations 2019BAAS...51c.144F Furlanetto
% 3 Astro2020 Science White Paper: Synergies Between Galaxy Surveys and Reionization Measurements Furlanetto, 2019arXiv190306197F, 2019BAAS...51c.142F
% 4 Spectral Distortions of the CMB as a Probe of Inflation, Recombination, Structure Formation and Particle Physics 2019BAAS...51c.184C Chluba
% 5 Messengers from the Early Universe: Cosmic Neutrinos and Other Light Relics 2019BAAS...51c.159G Green
% 6 Astrophysics and Cosmology with Line-Intensity Mapping, Kovetz 2019BAAS...51c.101K
% 7 Astro2020 Science White Paper: First Stars and Black Holes at Cosmic Dawn with Redshifted 21-cm Observations Mirocha 2019arXiv190306218M
% 8 Primordial Non-Gaussianity Meerburg 2019BAAS...51c.107M
% 9 Inflation and Dark Energy from spectroscopy at z > 2 Ferraro 2019BAAS...51c..72F
% 10 Cosmic Dawn and Reionization: Astrophysics in the Final Frontier 2019BAAS...51c..48C Cooray
% 11 Science from an Ultra-Deep, High-Resolution Millimeter-Wave Survey Shegal 2019BAAS...51c..43S
% 12 CMB-HD: An Ultra-Deep, High-Resolution Millimeter-Wave Survey Over Half the Sky Sehgal 2019arXiv190610134S
% 13 Populations behind the source-subtracted cosmic infrared background anisotropies Kashlinsky 2019BAAS...51c..37K
% 14 Rich Barry's Whitepaper
% 15 LiteBIRD whitepaper
\newcommand{\Decadal}{Co-signer or contributor to 15 Decadal White Papers and 2 ESA Voyage 2050 White Papers (2019).}

\newcommand{\DecadalIM}{E.\ D.\ Kovetz, P.\ C.\ Breysse, A.\ Lidz, J.\ Bock,
C.\ M.\ Bradford, T.-C.\ Chang, S.\ Foreman, H.\ Padmanabhan, A.\ Pullen, D.\
Riechers, M.\ B.\ Silva, {\bf E.\ R.\ Switzer}, ``Astrophysics and Cosmology
with Line-Intensity Mapping," {\em Contribution to the 2020 Decadal Survey},
March 2019.}

% Talks
\newcommand{\GSFCAPRA}{``EXCLAIM and the APRA program" {\em PI Workshop}, NASA
GSFC, February 2020.} % Feb 28

\newcommand{\HQEXCLAIMone}{``The Experiment for Cryogenic Large-Aperture
Intensity Mapping (EXCLAIM)," {\em PI Workshop, in absentia by Emily
Barrentine}, NASA HQ, October 2019.} % Oct 7

\newcommand{\CITAEXCLAIM}{``EXCLAIM: a new balloon mission to map the
cosmological history of galaxies," {\em Seminar}, Canadian Institute for
Theoretical Astrophysics, University of Toronto, May 2019.} % May 9

\newcommand{\CCAIMEXCLAIM}{``The Experiment for Cryogenic Large-Aperture
Intensity Mapping (EXCLAIM)," {\em Intensity Mapping Workshop}, Center for
Computational Astrophysics, Simons Institute, February 2019.} % Feb 21

\newcommand{\IHPIM}{``Challenges in Analysis of Intensity Mapping Data," {\em
Analytics, Inference, and Computation in Cosmology}, Institut Henri Poincar\'e,
Paris, October 2018.} % Oct 25

\newcommand{\CataldoIAC}{Giuseppe Cataldo et al., ``Second-generation
Micro-Spec: a compact spectrometer for far-infrared and submillimeter space
missions," {\em Proc. IAC}(69th congress), 2018.} % Oct 2018

\newcommand{\AspenIM}{``Interpreting intensity mapping data in the presence of
foregrounds," {\em Cosmological Signals from Cosmic Dawn to the Present},
Aspen, February 2018.} % Feb 5

\newcommand{\BerkIM}{``Measurements, prospects and challenges for line
tomography after reionization," {\em Radio Astronomy Lab Seminar}, Berkeley,
August 2017.} % Aug 6

\newcommand{\RRIPIXIE}{``The Primordial Inflation Explorer (PIXIE)," {\em CMB
spectral distortions from cosmic baryon evolution}, Raman Research Institute,
July 2016.} % July 13

\newcommand{\RRIIMFG}{``Global foregrounds and intensity mapping," {\em CMB
spectral distortions from cosmic baryon evolution}, Raman Research Institute,
July 2016.} % July 13

\newcommand{\GBTIMSLAC}{``Cosmic tomography with the Green Bank Telescope,"
{\em Opportunities and Challenges in Intensity Mapping}, SLAC Workshop, Mar.
2016.} % March 22

\newcommand{\PIPERCADR}{``PIPER's Continuous Adiabatic Demagnetization
Refrigerator," {\em B-modes from Space}, IMPU Workshop, Dec. 2015.} % dec 15

\newcommand{\JHUIMPIPER}{``Cosmic tomography with the GBT and status of the
Primordial Inflation Polarization ExploreR (PIPER)," {\em JHU seminar}, Nov.
2015.} % Nov 9

%originally: catching a wave on the last big photosphere
\newcommand{\GWUBICEP}{``Seeking signs of cosmological inflation in the CMB
polarization: BICEP2 and efforts at GSFC," {\em GWU DC/MD/VA Astrophysics
Summer 2014 meeting}, July 2014.} % July 10

\newcommand{\SEDPIPER}{``The Primordial Inflation Polarization Explorer
(PIPER)," {\em Science and Exploration Directorate Director's Seminar (GSFC)},
Nov.  2013.} % Nov 8

\newcommand{\KITPGBT}{``Results from the Green Bank Telescope 21 cm intensity
survey," {\em Observations and Theoretical Challenges in Primordial Cosmology
(KITP)}, Apr. 2013.} % Apr. 26

\newcommand{\OhioGBT}{``Results from the Green Bank Telescope 21 cm intensity
survey: Methods," {\em Innovative Techniques in 21 cm Analysis (Ohio)}, Apr. 2013.} % Apr. 18
% http://www.ccapp.osu.edu/workshops/21cm/abstracts.html

\newcommand{\NASAjobtalk}{``A history of the universe through its atoms," {\em
NASA GSFC, cosmology division seminar}, Jan. 2013.} % Feb. 26

\newcommand{\CMUjobtalk}{``A history of the universe through its atoms," {\em
CMU, physics colloquium}, Jan. 2013.} % Jan. 22

\newcommand{\PennGBT}{``$21$~cm Intensity Mapping with the Green Bank
Telescope," {\em UPenn, astrophysics seminar}, Dec. 2012.} % Dec. 12

\newcommand{\PIPERAASStokesV}{``The Primordial Inflation Polarization ExploreR:
Science from Circular Polarization Measurements," {\em AAS 223}, Jan. 2014.}

% Mar 26
\newcommand{\BICEPlocal}{``BICEP2: detection of B-mode polarization at degree
angular scales," {\em presentation for GSFC Fermi group}, Mar. 2014.}

% http://cdsagenda5.ictp.trieste.it/full_display.php?ida=a11171
\newcommand{\ICTPGBT}{``$21$~cm Intensity Mapping with the Green Bank
Telescope," {\em Workshop on Recent Developments in Astronuclear and
Astroparticle Physics (ICTP)}, Nov. 2012.} % Nov. 22
% astroparticle and astronuclear 

\newcommand{\SCMAVproc}{E.\ R.\ Switzer, T.\ M.\ Crawford, C.\ L.\ Reichardt,
``Bayesian flux reconstruction in one and two bands," Statistical Challenges in
Modern Astronomy V, Feigelson and Babu (Eds.), Springer 2012.}
\newcommand{\SCMAVprochttp}[1]{\linkarticle{\SCMAVproc}{http://www.springer.com/statistics/book/978-1-4614-3519-8}{#1}}

\newcommand{\WaterlooGBT}{``$21$~cm Intensity Mapping with the Green Bank
Telescope," {\em University of Waterloo, astrophysics seminar}, Jan. 2012.} % Jan. 25

\newcommand{\cancunGBT}{``$21$~cm Intensity Mapping with the Green Bank
Telescope," {\em Cosmology on the Beach, Cancun}, Jan. 2012.} % Jan. 20

\newcommand{\GTWOKGBT}{``$21$~cm Intensity Mapping with the Green Bank
Telescope," {\em G2000, University of Toronto}, Nov. 2011.} % Nov. 23

\newcommand{\SCMAV}{``Bayesian flux reconstruction in one and two bands," {\em
Statistical Challenges in Modern Astronomy V (Penn State University)}, June
2011.} % June 13
\newcommand{\SCMAVhttp}[1]{\linkarticle{\SCMAV}{http://astrostatistics.psu.edu/su11scma5/scma5.html}{#1}}

\newcommand{\CITAfore}{``Removing foregrounds and characterizing residuals in
$z \sim 1$ 21 cm surveys," {\em 21-cm Cosmology: Advanced data analysis
(CITA)}, June 2011.} % June 7

\newcommand{\CITAJob}{``Some Aspects of Cosmological Helium," {\em CITA,
astrophysics seminar}, Jan. 2011.} % Jan. 24
\newcommand{\CITAJobhttp}[1]{\linkarticle{\CITAJob}{http://cita.utoronto.ca/~eswitzer/talks/helium_history_CITA.pdf}{#1}}

\newcommand{\FNALJob}{``Some Aspects of Cosmological Helium," {\em Fermilab
Center for Particle Astrophysics, seminar}, Dec. 2010.} % Dec. 7
\newcommand{\FNALJobhttp}[1]{\linkarticle{\FNALJob}{http://cita.utoronto.ca/~eswitzer/talks/Helium_Cosmology_ERS.pdf}{#1}}

\newcommand{\BerkeleySPT}{``Statistics of the source population observed at
millimeter wavelengths by the South Pole Telescope," {\em Berkeley,
astrophysics seminar}, Mar. 2010.} % Mar. 16
\newcommand{\BerkeleySPThttp}[1]{\linkarticle{\BerkeleySPT}{http://cita.utoronto.ca/~eswitzer/talks/Switzer_SPT_ptsrc_Berkeley_2010.pdf}{#1}}

\newcommand{\KICPSPT}{``Statistics of the source population observed at
millimeter wavelengths by the South Pole Telescope," {\em KICP Postdoctoral
Symposium}, Feb. 2010.} % Feb. 22

\newcommand{\KICPGBT}{``Prospects for cosmology at $z \sim 1$ with $21$~cm
radiation," {\em KICP Postdoctoral Symposium}, Mar. 2011.} % Mar. 11

\newcommand{\EPAenergy}{``A physicist's outlook on energy," {\em Environmental
Protection Agency Region 5 office, seminar}, Feb. 2010.}  % Feb. 3
\newcommand{\EPAenergyhttp}[1]{\linkarticle{\EPAenergy}{http://cita.utoronto.ca/~eswitzer/compton/}{#1}}

\newcommand{\ComptonLecture}[1]{\begin{samepage}
\linkarticle{``The Physics of Energy Devices," Compton Lectures, University of
Chicago, Fall 2009:}{http://cita.utoronto.ca/~eswitzer/compton/}{#1}
\begin{resitem}
\item Lecture 1: Introduction and motors
\item Lecture 2: Motors and generators
\item Lecture 3: Power transmission
\item Lecture 4: Power from the wind
\item Lecture 5: Basic thermodynamics
\item Lecture 6: Heat engines and transportation
\item Lecture 7: Nuclear fission
\item Lecture 8: Solar energy
\item Lecture 9: Special guest lecture -- Dorian Abbot
\item Lecture 10: Summary; future
\end{resitem}
\end{samepage}
}

\newcommand{\hyperfineradtrans}{``Radiative transport through the hyperfine
transitions," {\em KICP, theory group talk}, Dec. 2009.}  % Dec. 7
\newcommand{\hyperfineradtranshttp}[1]{\linkarticle{\hyperfineradtrans}{http://cita.utoronto.ca/~eswitzer/documents/radiative_transport_hyperfine.pdf}{#1}}

\newcommand{\energygroup}{Several informal presentations for the energy
technology study group, University of Chicago, 2009.}
\newcommand{\energygrouphttp}[1]{\linkarticle{\energygroup}{https://toaster.uchicago.edu/energy/index.php/Schedule}{#1}}

\newcommand{\refineries}{``Refining oil,'' {\em Energy technology study group,
University of Chicago}, Apr.\ 2009.}

\newcommand{\KICPhyperfine}{``Wandering in the hyperfine forest,'' {\em KICP
seminar, University of Chicago}, Apr.\ 2009.}

\newcommand{\solartheory}{``Introduction to Solar Cells and Efficiency
Arguments,'' {\em Energy technology study group, University of Chicago}, Mar.\
2009.}

\newcommand{\energygeneral}{``Energy: Present Use and Future Choices,''
{\em Energy technology study group, University of Chicago}, Jan.\ 2009.}

\newcommand{\thesistalk}{``Small-Scale Anisotropies of the CMB: Experimental
and Theoretical Perspectives,'' {\em Ph.D. Final Public Oral, Princeton}, Oct.
2008.}

\newcommand{\MPArecombinationtwo}{``Prospects for observing the spectral
distortion from recombination,'' {\em The Physics of Cosmological Recombination
(MPA),} July 2008.}

\newcommand{\MPArecombinationone}{``Cosmological helium recombination,'' {\em
The Physics of Cosmological Recombination (MPA),} July 2008.}

\newcommand{\SPIEtes}{Y.\ Zhao, ACT collaboration, ``Characterization of
Transition Edge Sensors for the Millimeter Bolometer Array Camera on the
Atacama Cosmology Telescope,'' {\em Proc.\ SPIE}, {\bf 7020}, 70200O-70200O-11
(2008).}
\newcommand{\SPIEteshttp}[1]{\linkarticle{\SPIEtes}{http://adsabs.harvard.edu/abs/2008SPIE.7020E..20Z}{#1}}

\newcommand{\SPIEopt}{R.\ J.\ Thornton, ACT collaboration, ``Opto-mechanical
design and performance of a compact three-frequency camera for the MBAC
receiver on the Atacama Cosmology Telescope,'' {\em Proc.\ SPIE}, {\bf 7020}
70201R-70201R-10 (2008).}
\newcommand{\SPIEopthttp}[1]{\linkarticle{\SPIEopt}{http://adsabs.harvard.edu/abs/2008SPIE.7020E..44T}{#1}}

\newcommand{\SPIEsys}{{\bf E.\ R.\ Switzer}, ACT collaboration, ``Systems and
control software for the Atacama Cosmology Telescope,'' {\em Proc.\ SPIE}, {\bf
7019} 70192L-70192L-12 (2008).}
\newcommand{\SPIEsyshttp}[1]{\linkarticle{\SPIEsys}{http://www.physics.princeton.edu/act/papers/switzer_spie_2008.pdf}{#1}}

\newcommand{\SPIEmbac}{D.\ S.\ Swetz, ACT collaboration, ``Instrument design
and characterization of the Millimeter Bolometer Array Camera on the Atacama
Cosmology Telescope,'' {\em Proc.\ SPIE}, {\bf 7020} 702008-702008-12 (2008).}
\newcommand{\SPIEmbachttp}[1]{\linkarticle{\SPIEmbac}{http://adsabs.harvard.edu/abs/2008SPIE.7020E...6S}{#1}}

\newcommand{\SPIEtele}{A.\ D.\ Hincks, ACT collaboration, ``The effects of the
mechanical performance and alignment of the Atacama Cosmology Telescope on the
sensitivity of microwave observations,'' {\em Proc.\ SPIE}, {\bf 7020}
70201P-70201P-10 (2008).}
\newcommand{\SPIEtelehttp}[1]{\linkarticle{\SPIEtele}{http://adsabs.harvard.edu/abs/2008SPIE.7020E..42H}{#1}}

\newcommand{\SPIEmce}{E.\ S.\ Battistelli, ACT collaboration, ``Automated SQUID
tuning procedure for kilo-pixel arrays of TES bolometers on the Atacama
Cosmology Telescope,'' {\em Proc. SPIE}, {\bf 7020} 702028-702028-12 (2008).}
\newcommand{\SPIEmcehttp}[1]{\linkarticle{\SPIEmce}{http://adsabs.harvard.edu/abs/2008SPIE.7020E..57B}{#1}}

\newcommand{\ggfour}{``Millimeter-wave emission of the planets,'' {\em
Princeton Gravity Group}, Feb., 2008.}

\newcommand{\postdoctalks}{``Small-scale CMB Anisotropies and the Atacama
Cosmology Telescope: Perspectives and Progress,'' {\em Talk for the Chicago,
Berkeley cosmology groups}, Oct. 2007.}

\newcommand{\postdoctalksberkeley}{``Small-scale CMB Anisotropies and the
Atacama Cosmology Telescope: Perspectives and Progress,'' {\em Berkeley,
cosmology group seminar}, Oct. 2007.}

\newcommand{\postdoctalkschicago}{``Small-scale CMB Anisotropies and the
Atacama Cosmology Telescope: Perspectives and Progress,'' {\em KICP,
seminar}, Oct. 2007.}

\newcommand{\actkilo}{M. Niemack, ACT collaboration, ``A kilopixel array of TES
bolometers for ACT: development, testing, and first light,'' {\em J. Low Temp.
Phys.}, {\bf 151}(3-4) 690-696 (2008).}
\newcommand{\actkilohttp}[1]{\linkarticle{\actkilo}{http://adsabs.harvard.edu/abs/2008JLTP..151..690N}{#1}}

\newcommand{\ggthree}{``Cosmological helium recombination,'' {\em Princeton
Gravity Group}, Feb., 2007.}

\newcommand{\gslhome}{E.\ R.\ Switzer, ``Graduate Student Life -- Home on the
Range,'' {\em Princeton physics departmental newsletter}, Sept.\ 2006.}

\newcommand{\aiguide}{E.\ R.\ Switzer, ``Physics AI Guide: Princeton
University,'' {\em Guide for graduate assistant instructorships}, Sept.\ 2006.}

\newcommand{\actprogress}{A.\ Kosowsky, the ACT collaboration, ``The Atacama
Cosmology Telescope: A progress report,'' {\em New Astronomy Reviews} {\bf
50}(11-12) 969-976 (2006); Switzer: {\em preliminary beam maps appearing
therein}.}
\newcommand{\actprogresshttp}[1]{\linkarticle{\actprogress}{http://adsabs.harvard.edu/abs/2006NewAR..50..969K}{#1}}

\newcommand{\niemackmmtes}{M.\ Niemack, the ACT collaboration, ``Measuring
two-millimeter radiation with a prototype multiplexed TES receiver for ACT,''
{\em Proc. SPIE} {\bf 6275} 62750C (2006).}

\newcommand{\ggtwo}{``Corrections to cosmological recombination,'' {\em
Princeton Gravity Group}, Apr.\ 2006.}

\newcommand{\advproj}{``Numerical radiative transport for recombination
physics,'' {\em Princeton advanced project under C.\ M.\ Hirata}, Apr.\ 2005.}

\newcommand{\ggone}{``Weren't we done with recombination?'' {\em Princeton
Gravity Group}, Mar.\ 2005.}

\newcommand{\expreport}{``Variance estimates in the SDSS spectrographic data,''
{\em Princeton experimental project}, Sept.\ 2004.}

\newcommand{\SDSSreport}{``Variance estimates in the SDSS spectrographic
data,'' {\em Sloan Digital Sky Survey internal report}, Sep.\ 2004.}

\newcommand{\neutrinolssproc}{{\bf E.\ R.\ Switzer}, K.\ Abazajian, S.\
Dodelson, S.\ Habib, and K.\ Heitmann, ``Massive neutrinos and the halo model
of large scale structure,'' {\em Nuc.\ Phys.\ B, Proceedings from Neutrino
2004} {\bf 143}, 571 (2005).}
\newcommand{\neutrinolssprochttp}[1]{\linkarticle{\neutrinolssproc}{http://adsabs.harvard.edu/abs/2005NuPhS.143..571S}{#1}}

\newcommand{\bathesis}{E.\ R.\ Switzer, ``OPAL/LEP II measurements of $\tau$
polarization at $\approx 206$~GeV," {\em Undergraduate thesis under Mark
Oreglia, University of Chicago} (2003).}

\newcommand{\hermes}{E.\ R.\ Switzer, ``Measurements of electron energy
deposition in the HERMES silicon recoil detector,'' {\em DESY HERMES Recoil
Group internal report}, Aug. 2002.}

\newcommand{\mucool}{E.\ R.\ Switzer, ``Thermal diffusion in the MuCool
bolometer,'' {\em EFI-HEP Muon Cooling Group internal report}, June 2002.}

\newcommand{\mucoolamp}{E.\ R.\ Switzer, ``MuCool calorimeter amplifier design
and analysis,'' {\em EFI-HEP Muon Cooling Group internal report}, July 2001.}

\newcommand{\sspferonia}{{\bf E.\ R.\ Switzer}, et al., Orbital elements of
asteroid 72-Feronia, submitted to Harvard small planet catalog. (1998).}

%------------------------------------------------------------------------------
% publication sections
%------------------------------------------------------------------------------
% argument gives character to use for the link
\newcommand{\preprints}[1]{
\reshead{Submitted Publications}
\begin{enumerate}
\item \ACTsixlensmaphttp{#1}
\item \ACTsixlenscosmohttp{#1}
\item \PIXIEsysthttp{#1}
\item \PIPERdetectorshttp{#1}
\end{enumerate}
}

\newcommand{\refereed}[1]{
\reshead{Refereed Publications}
\begin{enumerate}
\item \DESlensinghttp{#1}
\item \EXCLAIMforecasthttp{#1}
\item \LBOMThttp{#1}
\item \LBsenshttp{#1}
\item \LiteBIRDPTEPhttp{#1}
\item \OIIIxCIIhttp{#1}
\item \KIDoptimumhttp{#1}
\item \FIRASBOSShttp{#1}
\item \EXCLAIMdesignhttp{#1}
\item \IMCVevadehttp{#1}
\item \PIPERpumpshttp{#1}
\item \PIPERwindowshttp{#1}
\item \ACTfoursummaryhttp{#1}
\item \ACTfourpowerhttp{#1}
\item \SPTptsrcfinalhttp{#1}
\item \LitebirdLTDhttp{#1}
\item \EXCLAIMoverviewhttp{#1}
\item \SPTPolptsrchttp{#1}
\item \IMPlanckhttp{#1}
\item \PIPERRXhttp{#1}
\item \IACuSpechttp{#1}
\item \ACTpolptsrchttp{#1}
\item \OdegardCIBhttp{#1}
\item \IMcontinuahttp{#1}
\item \Parkescrosshttp{#1}
\item \DAREmethodhttp{#1}
\item \DARErevhttp{#1}
\item \ACTPolStwohttp{#1}
\item \PIXIEIMhttp{#1}
\item \GBTICAhttp{#1}
\item \CMBlikehttp{#1}
\item \VPMsysthttp{#1}
\item \IMmethodshttp{#1}
\item \ACTPollensinghttp{#1}
\item \ACTlensxoptlenshttp{#1}
\item \GlobalFGhttp{#1}
\item \ACTSZRadiohttp{#1}
\item \ACTPolEEhttp{#1}
\item \EEtoTTreionhttp{#1}
\item \VPMReviewhttp{#1}
\item \actfinalpwrhttp{#1}
\item \ACTtwobandptsrchttp{#1}
\item \SPTptsrctwohttp{#1}
\item \actfinalparamhttp{#1}
\item \actfinalbeamhttp{#1}
\item \actfinalszhttp{#1}
\item \actfinallikehttp{#1}
\item \GBTautopowerhttp{#1}
\item \preplanckhttp{#1}
\item \SZmomentshttp{#1}
\item \ACTSZrichnesshttp{#1}
\item \ACTcluseqhttp{#1}
\item \recombmodeshttp{#1}
\item \GBTxWiggleZhttp{#1}
\item \ACTdatahttp{#1}
\item \ACTskewnesshttp{#1}
\item \ACTkxQSOhttp{#1}
\item \ACTkSZxBOSShttp{#1}
\item \HeItranshttp{#1}
\item \ACTSZAhttp{#1}
\item \ACTPpowerhttp{#1}
\item \ACTxBLASThttp{#1}
\item \ACTcalibhttp{#1}
\item \ACTparamhttp{#1}
\item \ACTclustercataloghttp{#1}
\item \ACTLambdahttp{#1}
\item \ACTlenshttp{#1}
\item \ACTLRGhttp{#1}
\item \Steppenwolfhttp{#1}
\item \ACTinstrumenthttp{#1}
\item \ACTclustercosmohttp{#1}
\item \ACTptsrchttp{#1}
\item \ACTdaspwrhttp{#1}
\item \actbeamszhttp{#1}
\item \ACTopticalclusterhttp{#1}
\item \HeIforesthttp{#1}
\item \actpwrspechttp{#1}
\item \sptptsrcclusteringhttp{#1}
\item \sptptsrcmethodhttp{#1}
\item \sptptsrccataloghttp{#1}
\item \hethreeplushttp{#1}
\item \thesishttp{#1}
\item \recthreehttp{#1}
\item \rectwohttp{#1}
\item \reconehttp{#1}
\item \actopthttp{#1}
\item \lisupphttp{#1}
\item \neutrinolsshttp{#1}
\item \electrochemhttp{#1}
\end{enumerate}
}

\newcommand{\reports}[1]{
\reshead{Proceedings, Reports}
\begin{enumerate}
\item \EXCLAIMSPIETEH
\item \EXCLAIMSPIECVhttp{#1}
\item \EXCLAIMSPIEMR
\item \LiteBIRDMHFThttp{#1}
\item \LiteBIRDLFThttp{#1}
\item \LiteBIRDoverviewhttp{#1}
\item \EXCLAIMSPIEuspechttp{#1}
\item \EXCLAIMSPIEopticshttp{#1}
\item \EXCLAIMSPIEhttp{#1}
\item \spectralvoyagehttp{#1}
\item \lambdaprimerhttp{#1}
\item \Decadal
\item \DecadalIM{#1}
\item \higatewayhttp{#1}
\item \PIPERSPIEShttp{#1}
\item \CataldoIAC{#1}
\item \IMstatushttp{#1}
\item \ADRSATprochttp{#1}
\item \CMBSIVtechhttp{#1}
\item \PIPERSPIEBhttp{#1}
\item \PIPERSPIEAhttp{#1}
\item \SCMAVprochttp{#1}
\item \SPIEmcehttp{#1}
\item \SPIEtelehttp{#1}
\item \SPIEmbachttp{#1}
\item \SPIEsyshttp{#1}
\item \SPIEopthttp{#1}
\item \SPIEteshttp{#1}
\item \actkilohttp{#1}
\item \aiguide
\item \gslhome %
\item \actprogresshttp{#1}
\item \niemackmmtes %
\item \advproj %
\item \SDSSreport %
\item \neutrinolssprochttp{#1}
\item \bathesis
\item \hermes %
%\item \mucool %
%\item \mucoolamp{#1} %
%\item \sspferonia{#1} %
\end{enumerate}
}

\newcommand{\talks}[1]{
\reshead{Talks, posters}
\begin{enumerate}
\item \UWEXCLAIM
\item \SEDEXCLAIM
\item \PIPERAASMissionhttp{#1}
\item \PIPERAASFlighthttp{#1}
\item \PIPERAASDetectorhttp{#1}
\item \PIPERAASReceiverhttp{#1}
\item \EXCLAIMAASMissionhttp{#1}
\item \AspenIM
\item \RRIPIXIE
\item \RRIIMFG
\item \GBTIMSLAC
\item \PIPERCADR
\item \JHUIMPIPER
\item \GWUBICEP
\item \BICEPlocal
\item \PIPERAASStokesV
\item \SEDPIPER
\item \KITPGBT
\item \OhioGBT
\item \NASAjobtalk
\item \CMUjobtalk
\item \PennGBT
\item \ICTPGBT
\item \WaterlooGBT
\item \cancunGBT
\item \GTWOKGBT
\item \SCMAVhttp{#1}
\item \CITAfore
\item \KICPGBT
\item \CITAJobhttp{#1}
\item \FNALJobhttp{#1}
\item \BerkeleySPThttp{#1}
\item \KICPSPT
\item \EPAenergyhttp{#1}
\item \hyperfineradtranshttp{#1}
\item \KICPhyperfine
\item \energygrouphttp{#1}
\item \thesistalk
\item \MPArecombinationtwo
\item \MPArecombinationone
\item \ggfour
%\item \postdoctalks
\item \postdoctalksberkeley
\item \postdoctalkschicago
\item \ggthree
\item \ggtwo
\item \ggone
\end{enumerate}
}
%\item \refineries
%\item \solartheory
%\item \energygeneral

\newcommand{\invitedtalks}[1]{
\reshead{Invited Talks}
\begin{enumerate}
\item \HQEXCLAIM
\item \IMspace
\item \CHORDcorr
\item \SOFIAEXCLAIM
\item \UWEXCLAIM
\item \SEDEXCLAIM
\item \GSFCAPRA
\item \HQEXCLAIMone
\item \CITAEXCLAIM
\item \CCAIMEXCLAIM
\item \IHPIM
\item \BerkIM
\item \RRIPIXIE
\item \RRIIMFG
\item \GBTIMSLAC
\item \PIPERCADR
\item \JHUIMPIPER
\item \GWUBICEP
\item \SEDPIPER
\item \KITPGBT
\item \OhioGBT
\item \NASAjobtalk
\item \CMUjobtalk
\item \PennGBT
\item \WaterlooGBT
\item \SCMAVhttp{#1}
\item \CITAfore
\item \CITAJobhttp{#1}
\item \FNALJobhttp{#1}
\item \BerkeleySPThttp{#1}
\item \EPAenergyhttp{#1}
\item \ComptonLecture{#1}
\item \KICPhyperfine
\item \MPArecombinationtwo
\item \MPArecombinationone
%\item \postdoctalks
\item \postdoctalksberkeley
\item \postdoctalkschicago
\end{enumerate}
}

\newcommand{\uninvitedtalks}[1]{
\reshead{Contributed Talks, Posters}
\begin{enumerate}
\item \PIPERAASMissionhttp{#1}
\item \PIPERAASFlighthttp{#1}
\item \PIPERAASDetectorhttp{#1}
\item \PIPERAASReceiverhttp{#1}
\item \EXCLAIMAASMissionhttp{#1}
\item \AspenIM
\item \BICEPlocal
\item \PIPERAASStokesV
\item \ICTPGBT
\item \cancunGBT
\item \GTWOKGBT
\item \KICPGBT
\item \KICPSPT
\item \hyperfineradtranshttp{#1}
\item \energygrouphttp{#1}
\item \thesistalk
\item \ggfour
\item \ggthree
\item \ggtwo
\item \ggone
\end{enumerate}
}

